
% Plano do Projeto TCC1 - Professora Ana Cristina B. Kochem Vendramin (cristina@dainf.ct.utfpr.edu.br, criskochem@utfpr.edu.br)

\documentclass{normas-utf-tex_07_2012} %Estilo de Formato criado pelo Prof. Hugo Vieira Neto (hvieir@utfpr.edu.br)
%Se voc� ainda n�o conhece o latex, comece olhando o site do Prof. Hugo --> http://pessoal.utfpr.edu.br/hvieir/orient/

%\documentclass[twoside,openright]{normas-utf-tex} %openright = o capitulo comeca sempre em paginas impares

\usepackage[num,abnt-emphasize=bf,bibjustif,recuo=0cm, abnt-etal-cite=2]{abntcite} %configuracao das referencias bibliograficas.
\usepackage[brazil]{babel} % pacote portugues brasileiro
\usepackage[latin1]{inputenc} % pacote para acentuacao direta
\usepackage{amsmath,amsfonts,amssymb,ams} % pacote matematico
\usepackage[pdftex]{graphicx} % pacote grafico
\usepackage{times} % fonte times
\usepackage{a4wide}
\usepackage[a4paper]{geometry} %define papel a4...
\geometry{left=3cm,right=2cm,top=3cm,bottom=2cm} % ...e suas margens
\usepackage{tabularx,multirow,longtable} %pacotes para mesclar linhas/colunas; tabelas grandes
\usepackage{fancyhdr} % altera cabe�alhos
\usepackage[T1]{fontenc} % acentua��o direto no texto
\usepackage{ae} % �Almost European�: aumenta a qualidade do pdf final
\usepackage{rotating} % faz rota��es de tabelas e figuras
\usepackage{indentfirst} % tabula a primeira linha do par�grafo
\usepackage[hang,small,bf]{caption} % legendas; nome "tabela" ou "figura" em negrito
\usepackage{caption3}
\usepackage{setspace} % ajuste do espa�o entre-linhas
\usepackage{algorithm}
\usepackage{subfig}
\usepackage{float} % posicionamento das figuras
\usepackage{subfloat}
\usepackage{arydshln}
\makeatletter
\renewcommand{\ALG@name}{Algoritmo}
\usepackage{algorithmic}

\usepackage{listings}
\usepackage{color}
\usepackage{xcolor}

\lstdefinestyle{customc}{
  belowcaptionskip=1\baselineskip,
  breaklines=true,
  frame=L,
  xleftmargin=\parindent,
  language=C,
  showstringspaces=false,
  basicstyle=\footnotesize\ttfamily,
  keywordstyle=\bfseries\color{green!40!black},
  commentstyle=\itshape\color{purple!40!black},
  identifierstyle=\color{blue},
  stringstyle=\color{orange},
}

%para a hifenacao funcionar � necessario fazer a seguinte modificacao
%Clique em Iniciar -> Programas -> MikTeX -> MiKTeX options V� em Languages e marque a caixa portugu�s.

%hifenacoes customizadas
\hyphenation{mo-de-lar re-no-v�-veis re-pre-sen-tar re-pre-sen-ta-��o
  la-te-rais res-pec-ti-vos re-a-��o re-a-li-zan-do o-pe-ra-��es cons-tan-te
    di-fe-ren-tes tem-pe-ra-tu-ra ex-tre-mi-da-de ter-mo-di-n�-mi-ca
    trans-es-te-ri-fi-ca-��o In-ter-net com-pu-ta-cio-nal}

%%%% Comandos introduzidos para controlar as alteracoes no fonte%%%%%%%%%%%%%%%%%
% assigning colors to comments of each author
\usepackage{color}
\newcommand{\cris}[1]{\textcolor{blue}{#1}} %texto final
\newcommand{\crisC}[1]{[\textcolor{blue}{#1}]} % coment�rio entre colchetes

%%%%%%%%%%%%%%%%%%%%%%%%%%%%%%%%%%%%%%%%%%%%%%%%%%%%%%%%%%%%%%%%%%%%%%%%%%%%%%%%%%%%%%

% ---------- Preambulo ----------
\instituicao{Universidade Tecnol\'ogica Federal do Paran\'a} % nome da instituicao
\programa{Departamento Acad\^emico de Inform\'atica} % nome do programa ou departamento
\area{Curso de Engenharia de Computa\c{c}\~ao} % �rea ou curso

\documento{Trabalho de Conclus\~ao de Curso} % [Trabalho de Conclus\~ao de Curso] ou [Disserta\c{c}\~ao] ou [Tese]
\nivel{Gradua\c{c}\~ao} % [Gradua\c{c}\~ao], [Especializa\c{c}\~ao], [Mestrado] ou [Doutorado]
\titulacao{Engenheiro} % [Engenheiro], [Tecn\'ologo], [Bacharel], [Mestre] ou [Doutor]

\titulo{\MakeUppercase{Extra��o Autom�tica de Placas de Ve�culos usando
  acelera��o por GPU}} % titulo do trabalho em portugues
\title{\MakeUppercase{Automatic License Plate Extraction using GPU acceleration}} % titulo do trabalho em ingles

\autor{Francisco Delmar Kurpiel} % autor do trabalho
\autordois{Luis Guilherme Camargo Machado}
\autortres{Marcelo Teider Lopes}
\cita{KURPIEL, Francisco D.; CAMARGO, Luis G. M.; LOPES, Marcelo T.} % sobrenome (maiusculas), nome do autor do trabalho

\palavraschave{ALPR, extra��o de placas veiculares, GPU, T-HOG, vis\~ao computacional} %substituir pelas palavras-chave relacionadas ao seu tema de pesquisa
\keywords{ALPR, license plate extraction, GPU, T-HOG, computer vision} % incluir palavras-chave em ingl�s

\comentario{\UTFPRdocumentodata\ de Engenharia da Computa\c{c}\~ao, apresentado ao \UTFPRprogramadata\ da \ABNTinstituicaodata\ como requisito parcial para obten\c{c}\~ao do t\'itulo de ``\UTFPRtitulacaodata\ em Computa\c{c}\~ao''.} %\\ \'Area de Concentra\c{c}\~ao: \UTFPRareadata.}

\orientador[Orientador:]{Prof. Bogdan Tomoyuki Nassu} % <- no caso de orientadora, usar esta sintaxe
%\coorientador[Co-orientadora:]{} % <- no caso de co-orientadora, usar esta sintaxe

\local{Curitiba} % cidade
\data{2015} % ano

\setcounter{tocdepth}{4} % para numera��o das se��es
\setcounter{secnumdepth}{4}

\numberwithin{equation}{chapter} %
%\numberwithin{table}{chapter} %
%\numberwithin{figure}{chapter} %

\makeatletter  %% this is crucial
 \renewcommand\subsubsection{\@startsection{subsubsection}{3}{\z@}%
                        {-2\p@ \@plus -0.5\p@ \@minus -0.5\p@}%
                        %{8\p@ \@plus 4\p@ \@minus 4\p@}%     <-- this is copied from the subsection command
                        {2\p@ \@plus 1\p@ \@minus 1\p@}%     <-- this is copied from the subsection command
                        {\normalfont\normalsize\bfseries\boldmath
                         \rightskip=\z@ \@plus 8em
 \pretolerance=10000 }}
\makeatother   %% this is crucial

%---------- Inicio do Documento ----------
\begin{document}

\capa % geracao automatica da capa

\folhaderosto % geracao automatica da folha de rosto
% dedicat�ria (opcional)
%\begin{dedicatoria}
% Texto da dedicat\'oria.
%\end{dedicatoria}

% agradecimentos (opcional)
%\begin{agradecimentos}
% Texto dos agradecimentos.
%\end{agradecimentos}

% epigrafe (opcional)
%\begin{epigrafe}
%\end{epigrafe}

%resumo
\begin{resumo}

TODO: Reescrever, fazer s� em um par�grafo \\
Deixar para mais tarde (quando o trabalho estiver mais conclu�do)
Deve conter: \\
Introdu��o \\
Objetivos Gerais \\
M�todo \\
Desenvolvimento \\
Resultados

Esse projeto tem o objetivo de extrair a localiza��o de placas de ve�culos de 
um \emph{stream} de v�deo, utilizando um algoritmo originalmente desenvolvido
para identificar se uma imagem � ou n�o texto.
 
Extra��o � uma das v�rias etapas necess�rias para o processo de 
reconhecimento autom\'atico de placas ve�culares. Em um \emph{stream} de v�deo, 
extra�mos, para cada \emph{frame}, a localiza��o de qualquer n\'umero de placas 
de ve\'iculos que possam ser encontradas. 

Como o algoritmo utilizado para a extra��o \'e altamente paraleliz\'avel, pois 
suas opera��es podem ser aplicadas de maneira independente em diversas 
regi\~oes da imagem, propomos sua implementa\c{c}\~ao em uma 
\sigla{GPU}{Graphical Processing Unit}.

A implementa��o ser\'a 
desenvolvida como uma biblioteca em \emph{C++} com o objetivo de ser utilizada 
em um sistema completo de reconhecimento de placas. 


\end{resumo}

%abstract
\begin{abstract}

This project aims to extract the location of license plates in a video stream,
by using an algorithm originally designed to identify if an image is or isn't 
text.

Extraction is one of the necessary steps for Automatic License Plate
Recognition. On a video stream we extract, for each frame, the location of any
number of license plates that can be found.

As the algorithm we're using for extraction is highly parallelizable, as it's
operations can be applied independently on several regions of the image, we
propose to implement it on a GPU.

The implementation will be developed as a C++ library, in
order for it to be used in a complete ALPR system.

\end{abstract}

% listas que recomenda-se a partir de 5 elementos
%\listadefiguras % geracao automatica da lista de figuras
%Elaborado de acordo com a ordem apresentada no texto, com cada item designado por seu nome espec�fico, acompanhado do respectivo n�mero da p�gina.

\listadetabelas % geracao automatica da lista de tabelas
%Elaborado de acordo com a ordem apresentada no texto, com cada item designado por seu nome espec�fico, acompanhado do respectivo n�mero da p�gina.

\listadesiglas % geracao automatica da lista de siglas
%Constitu�da de uma rela��o alfab�tica das abreviaturas e siglas utilizadas no texto, seguido das palavras ou express�es correspondentes grafadas por extenso. Utilizada apenas se houver siglas.

%\listadesimbolos % geracao automatica da lista de simbolos
%Elaborado de acordo com a ordem apresentada no texto, seguido do significado correspondente. Utilizada apenas se houver s�mbolos.

% sumario
\sumario % geracao automatica do sumario

%---------- Inicio do Texto ----------

\setcounter{page}{9} % *** Necess�rio arrumar manualmente antes de imprimir



\chapter{Introdu��o}\label{cap:introducao}

\section{Contexto}

TODO:reescrever/remover?

\section{Objetivos}

\noindent \textbf{Objetivo Geral}

Implementar o descritor de imagens HOG em CPU e GPU e comparar as
implementa��es em tempo de execu��o e diferen�a de resultados.

\noindent \textbf{Objetivos Espec�fico}

TODO

\noindent \textbf{Escopo}

TODO

\section{Resultados Esperados}

Utilizar uma GPU para realizar a extra��o pode trazer uma s�rie de ganhos para
o sistema:

\subsection{Tecnol�gicos}

\noindent \textbf{Melhor taxa de \emph{frames}}

Se os algoritmos forem mais r�pidos, torna-se poss�vel analisar mais quadros 
por segundo em um sistema de v�deo. No melhor caso, torna-se poss�vel analizar
o sistema em tempo real.

\noindent \textbf{Imagens de maior Resolu��o}

Uma c�mera de 1080p possui, por quadro, cerca do dobro da quantidade de pixels 
do uma c�mera 720p. A acelera��o via GPU pode viabilizar o uso de uma c�mera de 
maior resolu��o, o que resulta em maior nitidez e uma maior taxa de acertos ou, 
em alguns casos, a substitui��o de duas c�meras por apenas uma de maior 
resolu��o.

%\subsection{Cient�ficos}

\subsection{Econ�micos}

\noindent \textbf{Redu��o de custo}

TODO:escrever melhor esse item

Dado um sistema que j� atende aos requisitos da sua aplica��o um sistema deste 
pode rodar em um equipamento de menor custo, reduzindo o custo total da 
solu��o.

\noindent \textbf{Menor consumo de energia}

Uma GPU integrada possui um consumo de energia muito baixo se comparado a um
processador CPU comum.

\noindent \textbf{Menor demanda de largura de banda}

Elimina��o da transmiss�o e processamento remoto do \emph{stream} de v�deo em 
troca do processamento local e transmiss�o apenas de dados processados ou 
sumarizados, conforme necessidades espec�ficas.

%\subsection{Sociais}

%\subsection{Ambientais}

%\section{Procedimentos metodol�gicos}

\section{Apresenta��o do documento}

Inicialmente ser� apresentado um breve levantamento de trabalhos relacionados
� extra��o de caracter�sticas utilizando GPUs. A seguir ser� formalizada a
teoria utilizada no desenvolvimento do trabalho, em dois cap�tulos: um sobre
processamento de imagens e o algoritmo descrito utilizado e outro sobre a
arquitetura da GPU e o modelo de programa��o utilizado para processamento
paralelo na GPU.

No cap�tulo 5 se encontra um resumo do trabalho desenvolvido, seguido de um
relat�rio dos testes realizados. Finalmente, encontram-se algumas considera��es
finais sobre o projeto, incluindo indica��es de trabalhos futuros.


\chapter{Levantamento Bibiografico}\label{cap:levbibliog}

\section{Localiza��o de Placas}

\cite{l2008a} faz um \emph{survey} em 2008 de v�rias t�cnicas de reconhecimento 
de placas de tr�nsito. Neste artigo, captura em tempo real com limite de
50ms por \emph{frame}, j� que a taxa de 20\sigla{fps}{\emph{frames}  por
segundo}(\emph{frames} por segundo) 
� suficiente para que n�o se perca ve�culos que estejam passando pela c�mera.
Este \emph{survey} encontrou 
sistemas que tipicamente fazem pr� processamento, alguns usando imagens 
bin�rias, alguns usando tons de cinza e outros usando cores.

Imagens bin�rias: Alguns sistemas localizam placas usando algoritmos de 
detec��o de bordas que, apesar de muito r�pidos, geram muitos falsos positivos. 
Um sistema usa an�lise de componentes conectados que analisa a geometria
dos elementos da imagem (dimens�es e �rea) para determinar se cada 
componente � uma placa de tr�nsito. O uso do operador Sobel � encontrado
em v�rios dos algor�tmos estudados devido �s suas propriedades de elimina��o
de ru�do e do uso relativamente baixo de CPU (\emph{Central Processing Unit}), 
permitindo que o processamento total fique dentro dos 50ms.

Imagens em tons de cinza: Em quantidade de aplica��es esta categoria possui o 
maior n�mero publica��es. V�rios algor�timos usam contagem do n�mero de 
varia��es abruptas de contraste em um eixo, tipicamente o horizontal, para 
localizar placas. Este algor�timo pode operar em uma a cada N linhas da 
imagem, sendo muito econ�mico em tempo de CPU,
por�m � simplista demais para operar em v�rios cen�rios. Processamento 
estat�stico de bordas pode ser usado focando nas letras para operar bem quando 
o contorno da placa n�o � claro. Uma abordagem hier�rquica foi proposta usando
\emph{quadtrees}.
Cada quadrante � dividido em \emph{quadtrees} se tiver bastante 
varia��o de contraste. Segmentos cont�guos s�o agrupados se o brilho deles for 
muito diferente ou muito pr�ximo. Cada segmento recebe um escore pelo seu
tamanho e pelo escore dos blocos que o comp�e. Os melhores 
\emph{strips} s�o selecionados. Este algoritmo � bastante robusto a condi��es 
de ilumina��o e tem boa taxa de acertos. Uso de janela deslisante, nas quais a 
m�dia e o desvio padr�o s�o calculados e usados diretamente contra um limiar 
tamb�m foram usados com sucesso. Transformada \emph{Wavelet} foi aplicada para 
localizar placas, mas o m�todo se mostrou muito sens�vel a varia��es de 
dist�ncias e das caracter�sticas das lentes.

Tentativas de uso de informa��es de cor foram realizadas para tirar vantagem
dos padr�es utilizadas nos diversos pa�ses, por�m n�o se 
mostravam est�veis em condi��es naturais de ilumina��o, pelo fato de que a 
impress�o das cores varia de acordo com a mesma. As tentativas envolvem 
desde classifica��o \emph{pixel}-a-\emph{pixel} das cores at� uso de l�gica 
\emph{fuzzy} e redes neurais, com taxas variadas de acerto, de 75\% a 98\%. 

O uso de ilumina��o infravermelha foram demonstradas como sendo 
capaz de produzir sistemas com taxa de acerto de 99.3\%. Baixo n�mero de 
\emph{pixels} foi demonstrado como tendo efeito negativo em todos os sistemas 
testados.

\cite{gilly2013survey} � um \emph{review} de 2013 das mesmas t�cnicas. Desta
forma, � poss�vel verificar como as t�cnicas evolu�ram. Um m�todo simples de
janela deslisante com soma das 
proje��es verticais e horizontais foi utilizado com taxa de acerto de 96.7\%. 
Componentes conectados continuam sendo utilizados. L�gica \emph{fuzzy} foi 
utilizada com resultados que se mostraram sens�veis ilumina��o, cor e demandam 
muito uso de CPU. Contagem de varia��es abruptas em um eixo continua sendo 
utilizada.

\section{Reconhecimento de Texto}

Em \cite{sharma2012recent} � feito um \emph{survey} das t�cnicas recentemente 
usadas para realizar reconhecimento de texto em v�deo. O processo � separado em 
cinco etapas: sele��o de \emph{frames} com texto (identifica-se um quadro do 
v�deo tem ou n�o texto); detec��o e localiza��o de texto (duas etapas com um 
n�vel de interdepend�ncia, localizam as regi�es da imagem que tem texto); 
extra��o e aprimoramento (envolve separar o texto em caracteres e aplicar 
t�cnicas para melhorar a qualidade da imagem do caractere extra�do, como, por 
exemplo, binariza��o); e finalmente reconhecimento de caracteres.

O \emph{survey} separa as t�cnicas usadas para detec��o e localiza��o em duas 
categorias: baseadas em regi�o e baseadas em textura. Na primeira s�o usadas 
informa��es como cor, bordas e componentes conexos. J� as baseadas em textura 
usam principalmente an�lise no dom�nio da frequ�ncia, em especial com 
transformada \emph{Wavelet}.
Para a etapa de detec��o de texto, � proposta em \cite{minetto2013t} uma 
varia��o do conjunto de caracter�sticas HOG(\emph{Histograms of Oriented
Gradients}), denominada T-HOG(\emph{Text-HOG}). No conjunto de caracter�sticas 
HOG, descrito em \cite{dalal2005histograms}, a imagem � dividida em c�lulas 
de 8x8 \emph{pixels}, e estas c�lulas s�o agrupadas em blocos 2x2, com
intersec��o entre os blocos, de maneira que cada c�lula aparece quatro
vezes no descritor final (exceto as 
c�lulas da borda da imagem), � feita normaliza��o de contraste local (apenas 
dentro do bloco) e � calculado, para cada c�lula, um histograma dos gradientes, 
agrupados por �ngulo e ponderados pela magnitude. O descritor HOG � composto 
por estes histogramas. J� o T-HOG divide a imagem verticalmente em N c�lulas 
(N = 3 traz resultados satisfat�rios, resultados n�o melhoram muito para 
valores maiores que 7) e n�o agrupa em blocos.

\section{Uso de GPU para Extra��o de Caracter�sticas}

Em \cite{ready2007gpu} foram comparadas uma implementa��o baseada em CPU e
uma baseada em GPU(\emph{Graphical Processing Unit} de um algor�tmo
de extra��o de caracter�sticas.
Os autores comparam a performance do c�digo em uma CPU 
Pentium IV, 3.2GHz com uma GPU Quadro NVS 285. Os resultados demonstram que a 
CPU, com uso de 90\% da sua capacidade, consegue 
seguir 40 caracter�sticas em tempo real enquanto que a GPU consegue seguir at� 
500 caracter�sticas em tempo real. 

Em \cite{park2008low} � feita uma compara��o da performance de um algoritmo de 
detec��o de bordas em tempo real sendo executado em uma CPU e em duas GPUs. A 
CPU utilizada foi uma 2.4 GHz Intel Core 2 e as GPUs utilizadas 
foram 8800GTS-512 e uma placa de v�deo \emph{onboard} 8600MGT do Apple 
MacBookPro. Tanto a implementa��o na CPU como na GPU, utilizaram os mesmos 
par�metros e estrutura para que as compara��es sejam as mais corretas 
poss�veis. Os resultados demonstram que a GPU \emph{onboard} obteve uma 
performance 3.15 vezes melhor do que a CPU. A segunda GPU
obteve uma performance 22.96 vezes melhor do que a CPU. Os autores demonstram
com os resultados que � poss�vel conseguir ganhos de performance significativos
ao se utilizar uma GPU para algoritmos pesados, contanto que o c�digo seja 
propriamente otimizado para tirar proveito da arquitetura de uma GPU.


\chapter{Algoritmos}

\section{Gradientes de Imagem}

\section{Histogram of Oriented Gradients - HOG}

\section{Text-HOG - T-HOG}


\chapter{Recursos de Hardware e Software}\label{cap:recursos}

\section{Kit de desenvolvimento Jetson TK1}

A placa embarcada Jetson TK1, produzida pela NVidia, foi escolhida para o
projeto devido ao seu poder de processamento, baixo consumo e custo de \$192,
valor relativamente baixo se comparado com as alternativas dispon�veis no
mercado.

A placa utiliza o \sigla{SoC}{System on Chip}(\emph{System on Chip}) Tegra K1,
um processador de 4 cores ARM Cortex A15 e possui uma GPU integrada com a
arquitetura Kepler contendo 192 cores Cuda separados fisicamente em tr�s blocos.
Cada bloco tem sua pr�pria cache que � compartilhada entre seus 64 cores e pode
ser utilizada para sincronizar as threads em execu��o. A placa possui uma
mem�ria RAM de 2GB com largura de 64-bits, sendo ela compartilhada com entre a
CPU e a GPU, n�o sendo necess�rio copiar os dados entre as partes, diferente de
uma GPU discreta que possui sua pr�pria mem�ria dedicada separada da CPU. A
mem�ria � grande o suficiente para nos permitir trabalhar com imagem e v�deo.

Juntamente com a placa, utilizamos o ambiente de desenvolvimento da NVidia,
maisespecificadamente, uma vers�o modificada do Eclipse que nos permite
codificar qualquer tipo de aplica��o para a placa e para GPUs.

%A programa��o em uma GPU difere da programa��o em uma CPU em diversos aspectos
%como, por exemplo, a GPU executa m�ltiplos kernels em paralelo e tem um acesso
%especial � uma regi�o separada de mem�ria. Devido a sua arquitetura paralela,�
%poss�vel executar uma �nica tarefa com centenas de threads CUDA de maneira
%independente, utilizando dados de entrada diferentes para cada uma dessas
%threads. Portanto, caso seja possivel dividir uma tarefa grande em um
%subconjunto de tarefas menores e independentes, a programa�ao CUDA se
%demonstra eficiente.

\section{CUDA}

CUDA, uma \emph{framework} desenvolvida pela NVidia, permite que utilizemos uma
GPU para processamento de prop�sito geral, tamb�m conhecida como GPGPU. O
objetivo � que aplica��es possam utilizar a arquitetura paralela e da velocidade
fornecidas poruma GPU para fins al�m de aplica��es gr�ficas.

A API nos permite organizar as threads em blocos e criar conjuntos de
blocos chamados grids. � comum utilizarmos um n�mero muito maior de blocos nos grids
do que os fisicamente dispon�veis na placa, portanto a API se encarrega de
realizar o escalonamento autom�tico das atividades entre os processadores e
blocos fisicamente existentes. 

Cada Thread e Bloco possui uma ID pr�pria, sendo possivel utilizar essa informa��o para que cada Thread
trabalhe com um conjunto de dados �nicos. A API permite tamb�m organizar as threads em blocos e grids de maneira 2D ou 3D,
ficando o ID da Thread ou Bloco como uma tupla. Em processamento de imagens, as tarefas s�o usualmente realizadas em matrizes de duas 
dimens�es, portanto utilizar uma organiza��o em 2D facilita a escrita de c�digo, com cada Thread processando um pixel da imagem.

Sendo poss�vel programar e executar centenas de threads de forma �nica, tamb�m
�prov�vel que existam milhares de erros e bugs � serem consertados. Como o
programa principal e cada thread s�o executados de forma independente, n�o h�
uma falha ou aviso quando algo est� errado. � necess�rio esperar que as threads
terminem suas execu��es e sincroniz�-las com o programa para que possamos
procurar o erro. A API possui fun��es que permitem extrair qual o �ltimo erro
encontrado na execu��o e assim podemos debuggar nossas aplica��es.

\section{OpenCV}

OpenCV � uma biblioteca de c�digo-aberto para aplica��es de processamento de
imagem. A biblioteca � disponibilizada gratuitamente e recebe suporte por parte
de sua comunidade. OpenCV possui uma vers�o dedicada para aplica��es em GPU e
tamb�m a NVidia disponibiliza otimiza��es especif�cas para o processador Tegra
K1.

\chapter{Implementa��o do HOG}

Foi decidido realizar a implementa��o do HOG como uma biblioteca, de maneira a 
facilitar a utiliza��o em projetos futuros. Foram inclusas na biblioteca uma
implementa��o em GPU e outra em CPU, com uma interface comum.

No restante deste cap�tulo primeiramente ser�o apresentadas as ferramentas de
\emph{software} e \emph{hardware} utilizadas, seguido da apresenta��o da
organiza��o da biblioteca. Ap�s isso, uma documenta��o breve de cada um
dos m�dulos e finalmente uma descri��o detalhada da implementa��o em GPU.

\section{Ferramentas utilizadas}

\subsection{Kit de desenvolvimento Jetson TK1}

A placa embarcada Jetson TK1, produzida pela NVidia, foi escolhida para o
projeto devido ao seu poder de processamento, baixo consumo e custo de \$192,
valor relativamente baixo se comparado com as alternativas dispon�veis no
mercado.

A placa utiliza o \sigla{SoC}{System on Chip}(\emph{System on Chip}) Tegra K1,
um processador de 4 cores ARM Cortex A15 e possui uma GPU integrada com a
arquitetura Kepler contendo 192 cores Cuda separados fisicamente em tr�s blocos.
Cada bloco tem sua pr�pria cache que � compartilhada entre seus 64 cores e pode
ser utilizada para sincronizar as threads em execu��o. A placa possui uma
mem�ria RAM de 2GB com largura de 64-bits, sendo ela compartilhada com entre a
CPU e a GPU, n�o sendo necess�rio copiar os dados entre as partes, diferente de
uma GPU discreta que possui sua pr�pria mem�ria dedicada separada da CPU. A
mem�ria � grande o suficiente para nos permitir trabalhar com imagem e v�deo.

Juntamente com a placa, foi utilizado o ambiente de desenvolvimento da NVidia,
maisespecificadamente, uma vers�o modificada do Eclipse que nos permite
codificar qualquer tipo de aplica��o para a placa e para GPUs.

\subsection{OpenCV}

OpenCV\footnote{http://opencv.org} � uma biblioteca de c�digo-aberto para aplica��es 
  de processamento de
imagem. A biblioteca � disponibilizada gratuitamente e recebe suporte por parte
de sua comunidade. Ela possui um m�dulo com algumas funcionalidades
implementadas para aplica��es em GPU e
a NVidia disponibiliza uma vers�o compilada com otimiza��es especif�cas para o 
processador Tegra K1, como parte do \emph{framework} CUDA.

Foi utilizada a vers�o 2.4.9 da biblioteca para desenvolvimento e a vers�o
otimizada da NVidia para execu��o na placa.

\subsection{TinyXML-2}

Para realizar a perman�ncia de configura��o do sistema, foi utilizada a
biblioteca TinyXML-2\footnote{http://grinninglizard.com/tinyxml2}, inclusa como
dois arquivos-fonte no projeto.

\subsection{Boost}

Alguns detalhes de implementa��o foram supridos pelo uso de algumas bibliotecas
do conjunto de bibliotecas Boost\footnote{http://boost.org}. Mais
especificamente, foram usadas as bibliotecas \emph{thread}, \emph{chrono},
  \emph{random} e \emph{filesystem} para, respectivamente, fazer chamadas
  ass�ncronas de fun��o, medir o tempo de execu��o de trechos de c�digo,
  gera��o eficiente de n�meros aleat�rios e automatiza��o de acesso aos
  arquivos do sistema para realiza��o dos testes.

\subsection{GitHub}

Foi utilizada a plataforma GitHub para realizar o versionamento de c�digo do
projeto. A biblioteca pode ser encontrada em: 
<https://github.com/EAPVA/GHoGLib>, o c�digo usado para testes em
<https://github.com/EAPVA/GHoGLib\_Tests> e um c�digo de exemplo utilizando a
biblioteca em <https://github.com/EAPVA/GHoGLib\_Example>.

\section{Organiza��o da biblioteca}

A biblioteca possui as seguintes classes e interfaces:

\begin{itemize}
  \item \textbf{HogDescriptor} - Implementa��o na CPU do HOG.
  \item \textbf{HogGPU} - Implementa��o na GPU do HOG, � uma subclasse de
  HogDescriptor.
  \item \textbf{Settings} - Funcionalidade de perman�ncia de par�metros de
  configura��o.
  \item \textbf{Utils} - Classe est�tica com m�todos utilit�rios.
  \item \textbf{IClassifier} - Interface para implementar um classificador.
  Para ser utilizada em trabalhos futuros.
\end{itemize}

Al�m disso, a biblioteca possui dois \emph{headers} especiais:

\begin{itemize}
  \item \textbf{HogCallbacks.inc} - Possui m�todos de \emph{callback}
  para serem usados
  nos m�todos ass�ncronos.
  \item \textbf{GHoGLibConstants.inc} - Cont�m constantes usadas no sistema.
\end{itemize}

A implementa��o da classe HogGPU est� dividida em quatro arquivos:

\begin{itemize}
  \item \textbf{HogGPU.h} - Possui a defini��o da classe
  \item \textbf{HogGPU.cu} - Possui a implementa��o da parte \emph{host} da
  classe.
  \item \textbf{HogGPU\_impl.cuh} - Define as fun��es \emph{kernel} usadas na
  classe.
  \item \textbf{HogGPU\_imph.cu} - Implementa as fun��es \emph{kernel} usadas.
\end{itemize}

\subsection{Hog Callbacks}

Algumas classes de \emph{callback} s�o definidas para serem utilizadas em
m�todos ass�ncronos:

\begin{itemize}
  \item \textbf{ImageCallback} - Retorna uma image processada.
  \item \textbf{GradientCallback} - Retorna duas matrizes, uma com as
  magnitudes dos gradientes e outra com as orienta��es.
  \item \textbf{DescriptorCallback} - Retorna um vetor cont�ndo o descritor
  de caracter�sticas.
  \item \textbf{ClassifyCallback} - Retorna o resultado de uma classifica��o
  (se a regi�o de uma imagem pertence ou n�o a uma classe.) Para ser usado em
  trabalhos futuros.
  \item \textbf{LocateCallback} - Retorna uma lista de ret�ngulos de objetos
  indentificados em uma imagem. Para ser usado em trabalhos futuros.
\end{itemize}

\subsection{Constantes}

\subsubsection{GHOG\_LIB\_STATUS}

Define os c�digos de erro usados na biblioteca.

\begin{itemize}
  \item \textbf{GHOG\_LIB\_STATUS\_OK} - Nenhum erro ocorreu.
  \item \textbf{GHOG\_LIB\_STATUS\_UNKNOWN\_ERROR} - Um erro desconhecido ocorreu.
\end{itemize}

\subsection{HogDescriptor}

A classe HogDescriptor possui uma interface com os seguintes m�todos p�blicos:

\subsection{Settings}

\subsection{Utils}

\subsection{IClassifier}



\chapter{Testes}

%\chapter{Gest�o}

\section{Planilha de Horas}

\section{Planejamento de Riscos}

\noindent{\textbf{Dano no \emph{kit} de desenvolvimento}}

O \emph{kit} de desenvolvimento pode ser danificado, inviabilizando o projeto.

\textbf{Impacto:} 0 - Cr�tico

\textbf{Probabilidade:} 1 - M�dia 

\noindent \textbf{Mitiga��o:}
\begin{itemize} 
  \item Adquirir \emph{kits} adicionais.
\end{itemize}

\textbf{Impacto atualizado:} 0 - Cr�tico

\textbf{Probabilidade atualizada:} 2 - Baixa

\vspace{7mm}

\noindent{\textbf{Desenvolvimento em Cuda complexo demais}}

O desenvolvimento de c�digo utilizando Cuda pode ser complexo demais,
dificultando a realiza��o do projeto no tempo previsto

\textbf{Impacto:} 0 - Cr�tico

\textbf{Probabilidade:} 2 - Baixa

\noindent \textbf{Mitiga��o:}
\begin{itemize} 
  \item Priorizar aloca��o de horas-homem para esta atividade.
  \item Alocar atividades relacionadas a Cuda para come�arem o mais cedo
  poss�vel.
  \item Pesquisar alternativas para trabalhar com processamento paralelo.
\end{itemize}

\textbf{Impacto atualizado:} 1 - Significativo

\textbf{Probabilidade atualizada:} 2 - Baixa

\vspace{7mm}

\noindent{\textbf{Atividades n�o relacionadas ao projeto ou imprevistos tomarem
tempo excessivo da equipe}}

Como os integrantes da equipe n�o trabalham em dedica��o exclusiva ao projeto,
tendo outras atividades acad�micas e profissionais, estas podem prejudicar o
tempo dispon�vel ao projeto.

\textbf{Impacto:} 1 - Significativo

\textbf{Probabilidade:} 1 - M�dia

\noindent \textbf{Mitiga��o:}
\begin{itemize} 
  \item Realizar atividades com a maior anteced�ncia poss�vel.
  \item Priorizar este projeto em rela��o a outras atividades acad�micas.
\end{itemize}

\textbf{Impacto atualizado:} 0 - Baixo

\textbf{Probabilidade atualizada:} 1 - M�dia

\vspace{7mm}
\noindent{\textbf{Um dos integrantes deixar a equipe}}

Um dos integrantes, por motivos de: dificuldades com hor�rio, desentendimento
com a equipe, doen�a ou falecimento, pode vir a deixar a equipe.

\textbf{Impacto:} 1 - Significativo

\textbf{Probabilidade:} 2 - Baixa

\noindent \textbf{Mitiga��o:}
\begin{itemize} 
  \item Permitir hor�rios flex�veis para que nenhum integrante tenha que sair
  por problema de tempo.
  \item Comunica��o e reuni�es frequentes.
  \item Manter um ambiente de trabalho harm�nico.
  \item Evitar a concentra��o de conhecimento em um dos membros da equipe.
\end{itemize}

\textbf{Impacto atualizado:} 2 - Baixo

\textbf{Probabilidade atualizada:} 2 - Baixa

\vspace{7mm}

\noindent{\textbf{O \emph{kit} de desenvolvimento n�o ser r�pido o suficiente
  para rodar em tempo real}}

O \emph{kit} de desenvolvimento pode n�o possuir poder de processamento
suficiente para executar o algoritmo em tempo real (um \emph{frame} a cada
50ms). Sendo um projeto de pesquisa, se considerou que o impacto dessa falha �
baixo.

\textbf{Impacto:} 2 - Baixo

\textbf{Probabilidade:} 1 - M�dia 

\noindent \textbf{Mitiga��o:}
\begin{itemize} 
  \item Nenhuma estrat�gia de mitiga��o conhecida. 
\end{itemize}

\textbf{Impacto atualizado:} 2 - Baixo

\textbf{Probabilidade atualizada:} 1 - M�dia 
\vspace{7mm}

\noindent{\textbf{Treinamento do \emph{T-HOG} gerar baixa taxa de acertos}}

O algoritmo proposto pode n�o apresentar uma performance suficiente. Sendo um
projeto de pesquisa, se considerou que o impacto dessa falha � baixo.

\textbf{Impacto:} 2 - Baixo

\textbf{Probabilidade:} 2 - Baixa

\noindent \textbf{Mitiga��o:}
\begin{itemize} 
  \item Alocar um tempo razo�vel para fazer o \emph{tuning} do algoritmo 
\end{itemize}

\textbf{Impacto atualizado:} 2 - Baixo

\textbf{Probabilidade atualizada:} 2 - Baixa


\section{Considera��es}


\chapter{Conclus�o}\label{cap:conclusao}

\begin{enumerate}
\item Localiza��o das placas � um grande consumidor de tempo de
	processamento em sistemas de reconhecimento de ve�culos;
\item Sistemas de reconhecimento placas s�o tecnicamente vi�veis;
\item j� s�o aplicadas em outros pa�ses;
\item a tecnologia tem potencial para causar redu��o de custos ao substituir
	outros sistemas;
\item existe interesse governamental para fins de planejamento e fiscaliza��o;
\item Sistemas modernos de computa��o m�vel possuem GPU
\end{enumerate}

Com base nestes fatos acreditamos que:
com software suficientemente otimizado
� poss�vel produzir software de reconhecmento de ve�culos usando plataforma
de baixo custo;
a etapa de localiza��o das placas na imagem � um excelente
candidato a otimiza��o via delega��o para GPU;
o tema de localiza��o de placas em uma stream de v�deo seja pertinente
academicamente e realiz�vel, data a capacidade financeira e disponibilidade
de tempo dos autores deste trabalho.





%---------- Referencias ----------

\bibliography{Referencias}




\end{document}
