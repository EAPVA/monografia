
% Plano do Projeto TCC1 - Professora Ana Cristina B. Kochem Vendramin (cristina@dainf.ct.utfpr.edu.br, criskochem@utfpr.edu.br)

\documentclass{normas-utf-tex_07_2012} %Estilo de Formato criado pelo Prof. Hugo Vieira Neto (hvieir@utfpr.edu.br)
%Se voc� ainda n�o conhece o latex, comece olhando o site do Prof. Hugo --> http://pessoal.utfpr.edu.br/hvieir/orient/

%\documentclass[twoside,openright]{normas-utf-tex} %openright = o capitulo comeca sempre em paginas impares

\usepackage[num,abnt-emphasize=bf,bibjustif,recuo=0cm, abnt-etal-cite=2]{abntcite} %configuracao das referencias bibliograficas.
\usepackage[brazil]{babel} % pacote portugues brasileiro
\usepackage[latin1]{inputenc} % pacote para acentuacao direta
\usepackage{amsmath,amsfonts,amssymb,ams} % pacote matematico
\usepackage[pdftex]{graphicx} % pacote grafico
\usepackage{times} % fonte times
\usepackage{a4wide}
\usepackage[a4paper]{geometry} %define papel a4...
\geometry{left=3cm,right=2cm,top=3cm,bottom=2cm} % ...e suas margens
\usepackage{tabularx,multirow,longtable} %pacotes para mesclar linhas/colunas; tabelas grandes
\usepackage{fancyhdr} % altera cabe�alhos
\usepackage[T1]{fontenc} % acentua��o direto no texto
\usepackage{ae} % �Almost European�: aumenta a qualidade do pdf final
\usepackage{rotating} % faz rota��es de tabelas e figuras
\usepackage{indentfirst} % tabula a primeira linha do par�grafo
\usepackage[hang,small,bf]{caption} % legendas; nome "tabela" ou "figura" em negrito
\usepackage{caption3}
\usepackage{setspace} % ajuste do espa�o entre-linhas
\usepackage{algorithm}
\usepackage{subfig}
\usepackage{float} % posicionamento das figuras
\usepackage{subfloat}
\usepackage{arydshln}
\makeatletter
\renewcommand{\ALG@name}{Algoritmo}
\usepackage{algorithmic}

%para a hifenacao funcionar � necessario fazer a seguinte modificacao
%Clique em Iniciar -> Programas -> MikTeX -> MiKTeX options V� em Languages e marque a caixa portugu�s.

%hifenacoes customizadas
\hyphenation{mo-de-lar re-no-v�-veis re-pre-sen-tar re-pre-sen-ta-��o
  la-te-rais res-pec-ti-vos re-a-��o re-a-li-zan-do o-pe-ra-��es cons-tan-te
    di-fe-ren-tes tem-pe-ra-tu-ra ex-tre-mi-da-de ter-mo-di-n�-mi-ca
    trans-es-te-ri-fi-ca-��o In-ter-net com-pu-ta-cio-nal}

%%%% Comandos introduzidos para controlar as alteracoes no fonte%%%%%%%%%%%%%%%%%
% assigning colors to comments of each author
\usepackage{color}
\newcommand{\cris}[1]{\textcolor{blue}{#1}} %texto final
\newcommand{\crisC}[1]{[\textcolor{blue}{#1}]} % coment�rio entre colchetes

%%%%%%%%%%%%%%%%%%%%%%%%%%%%%%%%%%%%%%%%%%%%%%%%%%%%%%%%%%%%%%%%%%%%%%%%%%%%%%%%%%%%%%

% ---------- Preambulo ----------
\instituicao{Universidade Tecnol\'ogica Federal do Paran\'a} % nome da instituicao
\programa{Departamento Acad\^emico de Inform\'atica} % nome do programa ou departamento
\area{Curso de Engenharia de Computa\c{c}\~ao} % �rea ou curso

\documento{Trabalho de Conclus\~ao de Curso} % [Trabalho de Conclus\~ao de Curso] ou [Disserta\c{c}\~ao] ou [Tese]
\nivel{Gradua\c{c}\~ao} % [Gradua\c{c}\~ao], [Especializa\c{c}\~ao], [Mestrado] ou [Doutorado]
\titulacao{Engenheiro} % [Engenheiro], [Tecn\'ologo], [Bacharel], [Mestre] ou [Doutor]

\titulo{\MakeUppercase{Extra��o Autom�tica de Placas de Ve�culos usando
  acelera��o por GPU}} % titulo do trabalho em portugues
\title{\MakeUppercase{Automatic License Plate Extraction using GPU acceleration}} % titulo do trabalho em ingles

\autor{Francisco Delmar Kurpiel} % autor do trabalho
\autordois{Luis Guilherme Camargo Machado}
\autortres{Marcelo Teider Lopes}
\cita{KURPIEL, Francisco D.; CAMARGO, Luis G. M.; LOPES, Marcelo T.} % sobrenome (maiusculas), nome do autor do trabalho

\palavraschave{ALPR, extra��o de placas veiculares, GPU, T-HOG, vis\~ao computacional} %substituir pelas palavras-chave relacionadas ao seu tema de pesquisa
\keywords{ALPR, license plate extraction, GPU, T-HOG, computer vision} % incluir palavras-chave em ingl�s

\comentario{\UTFPRdocumentodata\ de Engenharia da Computa\c{c}\~ao, apresentado ao \UTFPRprogramadata\ da \ABNTinstituicaodata\ como requisito parcial para obten\c{c}\~ao do t\'itulo de ``\UTFPRtitulacaodata\ em Computa\c{c}\~ao''.} %\\ \'Area de Concentra\c{c}\~ao: \UTFPRareadata.}

\orientador[Orientador:]{Prof. Bogdan Tomoyuki Nassu} % <- no caso de orientadora, usar esta sintaxe
%\coorientador[Co-orientadora:]{} % <- no caso de co-orientadora, usar esta sintaxe

\local{Curitiba} % cidade
\data{2015} % ano

\setcounter{tocdepth}{4} % para numera��o das se��es
\setcounter{secnumdepth}{4}

\numberwithin{equation}{chapter} %
%\numberwithin{table}{chapter} %
%\numberwithin{figure}{chapter} %

\makeatletter  %% this is crucial
 \renewcommand\subsubsection{\@startsection{subsubsection}{3}{\z@}%
                        {-2\p@ \@plus -0.5\p@ \@minus -0.5\p@}%
                        %{8\p@ \@plus 4\p@ \@minus 4\p@}%     <-- this is copied from the subsection command
                        {2\p@ \@plus 1\p@ \@minus 1\p@}%     <-- this is copied from the subsection command
                        {\normalfont\normalsize\bfseries\boldmath
                         \rightskip=\z@ \@plus 8em
 \pretolerance=10000 }}
\makeatother   %% this is crucial

%---------- Inicio do Documento ----------
\begin{document}

\capa % geracao automatica da capa

\folhaderosto % geracao automatica da folha de rosto
% dedicat�ria (opcional)
%\begin{dedicatoria}
% Texto da dedicat\'oria.
%\end{dedicatoria}

% agradecimentos (opcional)
%\begin{agradecimentos}
% Texto dos agradecimentos.
%\end{agradecimentos}

% epigrafe (opcional)
%\begin{epigrafe}
%\end{epigrafe}

%resumo
\begin{resumo}

TODO: Reescrever, fazer s� em um par�grafo \\
Deixar para mais tarde (quando o trabalho estiver mais conclu�do)
Deve conter: \\
Introdu��o \\
Objetivos Gerais \\
M�todo \\
Desenvolvimento \\
Resultados

Esse projeto tem o objetivo de extrair a localiza��o de placas de ve�culos de 
um \emph{stream} de v�deo, utilizando um algoritmo originalmente desenvolvido
para identificar se uma imagem � ou n�o texto.
 
Extra��o � uma das v�rias etapas necess�rias para o processo de 
reconhecimento autom\'atico de placas ve�culares. Em um \emph{stream} de v�deo, 
extra�mos, para cada \emph{frame}, a localiza��o de qualquer n\'umero de placas 
de ve\'iculos que possam ser encontradas. 

Como o algoritmo utilizado para a extra��o \'e altamente paraleliz\'avel, pois 
suas opera��es podem ser aplicadas de maneira independente em diversas 
regi\~oes da imagem, propomos sua implementa\c{c}\~ao em uma 
\sigla{GPU}{Graphical Processing Unit}.

A implementa��o ser\'a 
desenvolvida como uma biblioteca em \emph{C++} com o objetivo de ser utilizada 
em um sistema completo de reconhecimento de placas. 


\end{resumo}

%abstract
\begin{abstract}

This project aims to extract the location of license plates in a video stream,
by using an algorithm originally designed to identify if an image is or isn't 
text.

Extraction is one of the necessary steps for Automatic License Plate
Recognition. On a video stream we extract, for each frame, the location of any
number of license plates that can be found.

As the algorithm we're using for extraction is highly parallelizable, as it's
operations can be applied independently on several regions of the image, we
propose to implement it on a GPU.

The implementation will be developed as a C++ library, in
order for it to be used in a complete ALPR system.

\end{abstract}

% listas que recomenda-se a partir de 5 elementos
%\listadefiguras % geracao automatica da lista de figuras
%Elaborado de acordo com a ordem apresentada no texto, com cada item designado por seu nome espec�fico, acompanhado do respectivo n�mero da p�gina.

\listadetabelas % geracao automatica da lista de tabelas
%Elaborado de acordo com a ordem apresentada no texto, com cada item designado por seu nome espec�fico, acompanhado do respectivo n�mero da p�gina.

\listadesiglas % geracao automatica da lista de siglas
%Constitu�da de uma rela��o alfab�tica das abreviaturas e siglas utilizadas no texto, seguido das palavras ou express�es correspondentes grafadas por extenso. Utilizada apenas se houver siglas.

%\listadesimbolos % geracao automatica da lista de simbolos
%Elaborado de acordo com a ordem apresentada no texto, seguido do significado correspondente. Utilizada apenas se houver s�mbolos.

% sumario
\sumario % geracao automatica do sumario

%---------- Inicio do Texto ----------

\setcounter{page}{9} % *** Necess�rio arrumar manualmente antes de imprimir



\chapter{Introdu��o}\label{cap:introducao}

O aumento do n�mero de ve�culos em cidades implica em v�rios desafios de 
planejamento urbano. Os problemas a serem resolvidos crescem tanto em escala
quanto em complexidade.

Em cidades maiores � preciso dimensionar vias e rotas para comportar os fluxos 
de cada regi�o. � necess�rio obter informa��o sobre origem e destino de cada 
ve�culo para determinar a utiliza��o das vias, e assim viabilizar o 
planejamento. Fiscaliza��o de velocidade, de convers�es ilegais e licenciamento 
dos ve�culos s�o tarefas fundamentais que se tornam invi�veis de serem feitas 
manualmente em larga escala.

Pode-se tamb�m pensar em tratamento de congestionamentos. Em algumas cidades, 
onde h� rod�zio de ve�culos baseados em n�mero de placa, � necess�rio um 
sistema capaz de fiscalizar uma grande quantidade de ve�culos, idealmente 
todos. Sistemas de tarifa��o de congestionamento, como o aplicado em Londres
\cite{londres}\cite{londres2}
, envolvem identificar todos os ve�culos em uma regi�o e 
emitir taxas, para desestimular o uso de ve�culos nestas �reas.

Reconhecimento autom�tico de placas � um processo que pode ser utilizado para 
tratar esses problemas. Tipicamente isso � realizado sobre imagem de v�deo 
e envolve v�rias etapas, sendo que uma delas, denominada extra��o, � 
respons�vel por localizar as posi��es das placas na imagem.

Muitos algoritmos de 
vis�o computacional s�o altamente paraleliz�veis, por operarem em dados de
forma iterativa e independente. A arquitetura das GPUs foi desenvolvida para
maximizar o \emph{throughput} desse tipo de algoritmo. Os resultados em 
\cite{ready2007gpu} e \cite{park2008low} demonstram um ganho consider�vel de 
performance ao se utilizar a GPU para atividades relacionadas com imagem 
e v�deo, ao inv�s de simplesmente utilizar a CPU. 

No entanto, para se obter tais ganhos deve-se utilizar a GPU de forma 
eficiente, sendo necess�rio entender sua estrutura e funcionamento para 
conseguir bons resultados. Se comparada com uma CPU, a GPU tamb�m possui 
limita��es que devem ser levadas em considera��o como, por exemplo, menor 
\emph{clock} e \emph{overhead} de transfer�ncia de dados.

\section{Contexto}

Existe um projeto em desenvolvimento na \sigla{UTFPR}{Universidade Tecnol�gica 
Federal do Paran�} (Universidade Tecnol�gica Federal do Paran�) que visa 
utilizar imagem de v�deo para medir a velocidade de ve�culos, reduzindo os 
custos de instala��o de sensores na via. Uma das etapas necess�rias � extra��o
de placas, que s�o usadas para rastrear os ve�culos. A atual implementa��o 
utiliza o algoritmo T-HOG em uma CPU Intel Core i7, n�o atingindo os requisitos 
de tempo real de 50ms por \emph{frame} \cite{l2008a}. 

Este trabalho prop�e implementar o algoritmo T-HOG em \sigla{CUDA}{Compute 
Unified Device Architecture} (\emph{Compute Unified Device Architecture}) para 
ser executado em uma GPU, de forma a melhorar a performance e possivelmente 
atingir os requisitos de tempo real. 

\section{Objetivos}

\noindent \textbf{Objetivo Geral}

Realizar uma implementa��o do algoritmo T-HOG em GPU para localizar placas de
ve�culos em um \emph{stream} de v�deo.

\noindent \textbf{Objetivo Espec�fico}

Implementar o algoritmo T-HOG usando \sigla{API}{Application Programming 
Interface} (\emph{Application Programming Interface}) Cuda da NVIDIA no 
\emph{kit} de desenvolvimento Jetson do mesmo fornecedor. A implementa��o ser� 
feita como uma biblioteca em C++, de forma que possa ser utilizado por 
qualquer \emph{software} que queira implementar as outras etapas do 
reconhecimento de placas.

\noindent \textbf{Escopo}

Das diversas etapas do reconhecimento de placas de tr�nsito, este projeto trata 
apenas da extra��o. Este processo inclui apenas indicar quantas placas foram 
encontradas e sua localiza��o. 
Estas informa��es ser�o disponibilizadas continuamente via
\sigla{API}{Application Programming Interface}(\emph{Application Programming
Interface}) da biblioteca em C++.

O escopo deste projeto n�o inclui etapas posteriores � extra��o, como 
tratamento de imagem, segmenta��o de d�gitos e reconhecimento dos caracteres 
constantes na mesma. 

\section{Resultados Esperados}

Utilizar uma GPU para realizar a extra��o pode trazer uma s�rie de ganhos para
o sistema:

\noindent \textbf{Melhor taxa de \emph{frames}}

Se os algoritmos forem mais r�pidos torna-se poss�vel analisar mais quadros por 
segundo. Em situa��es nas quais os ve�culos podem trafegar em alta velocidade, 
como em rodovias, � prefer�vel que as imagens sejam analizadas em tempo real,
onde a utiliza��o de uma GPU pode gerar ganhos consider�veis de performance.

\noindent \textbf{Possibilidade de CPU de menor custo}

Dado um sistema que j� atende aos requisitos da sua aplica��o um sistema deste 
pode rodar em um equipamento de menor custo, reduzindo o custo total da 
solu��o.

\noindent \textbf{Menor consumo de energia}

Dado que, em alguns casos, este tipo de sistema pode rodar, por exemplo, em um 
ve�culo da pol�cia, o consumo de energia do sistema precisa ser controlado para 
n�o exceder a capacidade de gera��o de energia do mesmo.
 
\noindent \textbf{Imagens de maior Resolu��o}

Uma c�mera de 1080p possui, por quadro, cerca do dobro da quantidade de pixels 
do uma c�mera 720p. A acelera��o via GPU pode viabilizar o uso de uma c�mera de 
maior resolu��o, o que resulta em maior nitidez e uma maior taxa de acertos ou, 
em alguns casos, a substitui��o de duas c�meras por apenas uma de maior 
resolu��o.

\noindent \textbf{Processamento em tempo real}

Algumas solu��es armazenam as imagens para serem analisadas em lotes, devido � 
incapacidade de tratamento destas imagens em tempo real. Isso elimina qualquer 
oportunidade que demande identifica��o da placa no momento da filmagem.

\noindent \textbf{Menor demanda de largura de banda}

Elimina��o da transmiss�o e processamento remoto do \emph{stream} de v�deo em 
troca do processamento local e transmiss�o apenas de dados processados ou 
sumarizados, conforme necessidades espec�ficas.


\chapter{Trabalhos Relacionados}\label{cap:levbibliog}

\section{Uso de GPU para Extra��o de Caracter�sticas}

Em \cite{zolynski2008lbpGPU} � descrita a implementa��o de um descritor baseado em
\emph{Local Binary Patterns}, usando o \emph{framework} Cg da NVIDIA, um
precursor de \sigla{GPGPU}{General Purpose Computing on Graphics Processing
  Units}(\emph{General-Purpose computing on Graphics Processing Units} -
         computa��o de prop�sito geral em unidades de processamento gr�fico)
  focado em reutilizar o processamento gr�fico de outras
maneiras. Comparando a performance de uma CPU Core2Quad 2.4GHz com as GPUs
GeForce 7600 GT, GeForce 8600 GT e GeForce 8800 GTS, foram obtidos tempos
de execu��o de aproximadamente ${{1} \over {14}}$ a ${{1} \over {18}}$ do tempo
gasto na CPU.

Em \cite{park2008low} � feita uma compara��o do desempenho de velocidade do algoritmo de 
detec��o de bordas de Canny aplicado em conjunto com o algoritmo \emph{Vector Coherence Mapping},
utilizado para detec�ao de movimentos em v�deos. Os algoritmos foram executados 
em uma CPU, 2.4 GHz Intel Core 2,  e duas GPUs, uma GeForce 8800GTS e uma placa de v�deo
\emph{onboard} 8600MGT do Apple MacBookPro. Tanto a implementa��o na CPU como 
na GPU utilizaram os mesmos par�metros e estrutura para que as compara��es 
fossem as mais corretas poss�veis. Os resultados demonstram que a GPU \emph{onboard}
 obteve uma velocidade de execu��o 3.15 vezes maior do que a CPU. A segunda GPU
obteve uma velocidade de execu��o 22.96 vezes maior do que a CPU.

\section{Implementa��es paralelas do HOG}

Em \cite{fastHOG} � descrita uma implementa��o multiescala do HOG em GPU. A implementa��o
foi testada em uma placa de v�deo GeForce GTX 285, obtendo velocidades de
execu��o at� 67 vezes maiores do que numa CPU. Para fazer o redimensionamento
necess�rio para a an�lise multiescala, eles copiam a imagem para o cache de
textura da placa, usando as fun��es de interpola��o em \emph{hardware}. A
normaliza��o de cor � executada em um bloco de tamanho $16 \times 16$, com cada
\emph{thread} processando um \emph{pixel}. Os gradientes s�o computados
usando dois \emph{kernels}, o primeiro computa os gradientes horizontais e o
segundo calcula os gradientes verticais, a magnitude e a orienta��o dos
gradientes. No c�lculo dos histogramas, no agrupamento em blocos e na
normaliza��o em blocos, cada bloco do HOG � mapeado para um bloco de
\emph{threads}.

Em \cite{hogfpga} � descrita uma implementa��o h�brida usando FPGA, CPU e GPU,
   alcan�ando tempo de execu��o de cerca de 300 microssegundos. O
   pr�-processamento, c�lculo de gradientes e histogramas � realizado na FPGA,
   a normaliza��o em blocos na CPU e o classificador SVM na GPU.

Em \cite{hogasic} � apresentado o desenvolvimento de um
\sigla{ASIC}{Application-Specific Integrated
  Circuit}(\emph{Application-Specific Integrated Circuit} - circuito integrado
           de aplica��o espec�fica) capaz de executar classifica��o multiescala
  usando HOG, 
  a uma taxa de 60 quadros por segundo em v�deo de alta defini��o (\emph{Full
                                                                   HD})

\chapter{Histogramas de gradientes orientados}

Neste cap�tulo s�o primeiramente definidos alguns conceitos b�sicos. Em
seguida, o algoritmo implementado nesse projeto � apresentado, o
\sigla{HOG}{Histogram of Oriented Gradients} (\emph{Histogram of Oriented
Gradients} - Hisgtogramas de Gradientes Orientados) \cite{dalal2005histograms}, 
  um descritor de
imagens que representa a distribui��o local das orienta��es dos gradientes.

\section{Imagem e \emph{Pixel}}

Para os prop�sitos desse trabalho, uma \emph{imagem} � definida como uma matriz
bidimensional $I$, sendo que $I[x,y]$ representa o elemento contido na
linha $y$ e coluna $x$ da matriz. Para respeitar a conven��o normalmente
utilizada, os �ndices $x$ e $y$ iniciar�o em $0$, e o \emph{pixel} $I[0,0]$ ser�
o canto superior esquerdo da imagem.

Cada elemento $I[x,y]$ de uma imagem � denominado de \emph{pixel}, sendo que
este pode conter informa��es de um ou mais \emph{canais de cor}. Cada canal de
cor cont�m um valor na faixa $[0,1]$, representando intensidade luminosa
naquele canal de cor. Caso a imagem possua apenas um canal de cor, ser� uma
imagem em escala de cinza. Neste
trabalho, ser�o usadas imagens em escala de cinza, contendo apenas um canal, e
imagens coloridas RGB, contendo os canais de cor vermelho, verde e azul.

\section{Gradientes de Imagem}

Para uma imagem em escala de cinza $I$, o gradiente de imagem � uma matriz $G$, 
com as mesmas dimens�es de $I$, onde cada ponto $G[x,y]$ representa a varia��o 
de luminosidade que ocorre em $I[x,y]$. Como a varia��o � bidimensional, cada
ponto do gradiente de imagem ser� um vetor de duas dimens�es, podendo ser
representado tanto na forma cartesiana quanto polar. Com base nisso, temos as
seguintes defini��es:

\emph{Gradiente horizontal} de uma imagem $I$ � o componente horizontal de $G$,
  sendo denotado por $G_x$ e calculado seguindo a equa��o
  \ref{eq:good_dx}:

\begin{equation} \label{eq:good_dx}
G_x[x,y] = I[x+1,y] - I[x-1,y]
\end{equation}

\emph{Gradiente vertical} de uma imagem $I$ � o componente vertical de $G$,
  sendo denotado por $G_y$ e calculado seguindo a equa��o
  \ref{eq:good_dy}:

\begin{equation} \label{eq:good_dy}
  G_y[x,y] = I[x,y+1]- I[x,y-1]
\end{equation}

\emph{Magnitude de gradiente} de uma imagem $I$ � a magnitude de $G$,
  sendo denotada por $G_{\rho}$ e calculada seguindo a equa��o
  \ref{eq:good_drho}:

\begin{equation} \label{eq:good_drho}
G_{\rho}[x,y] = \sqrt{G_x[x,y]^2 + G_y[x,y]^2}
\end{equation}

\emph{Orienta��o de gradiente} de uma imagem $I$ � a orienta��o (ou fase) de 
$G$, sendo denotada por $G_{\theta}$ e calculada seguindo a equa��o
  \ref{eq:good_dtheta}:

\begin{equation} \label{eq:good_dtheta}
G_{\theta}[x,y] = \arctan \left( {{G_y[x,y]} \over {G_x[x,y]}} \right)
\end{equation}

A magnitude do gradiente de um \emph{pixel} ter� valores pr�ximos de zero em
regi�es homog�neas da imagem e valores diferentes de zero nas regi�es de
transi��o de intensidade, ou seja, nas regi�es de borda.

Existem v�rias maneiras de obter o gradiente para uma imagem colorida. Ser�
utilizada a mesma escolhida em
\cite{dalal2005histograms}. Para cada \emph{pixel} ser�
calculado o gradiente de cada canal de cor e ser� escolhido o  que tiver a
maior magnitude.

\section{Histogram of Oriented Gradients - HOG}

O descritor de caracter�sticas HOG, descrito em \cite{dalal2005histograms}, 
  � composto por
diversos histogramas das orienta��es dos gradientes da imagem, calculados em 
uma parti��o da imagem em diferentes regi�es, chamadas de \emph{c�lulas}.
O c�lculo do HOG � dividido em uma s�rie de etapas: normaliza��o de cor da 
imagem, c�lculo dos gradientes, c�lculo dos histogramas e normaliza��o de 
contraste local.

\subsection{Normaliza��o}

Primeiramente � realizada uma normaliza��o de cor, substituindo o valor de cada
canal de cor de cada \emph{pixel} pela sua raiz quadrada.

\subsection{C�lculo dos gradientes}

Para cada cada um dos tr�s canais de cor da imagem, o gradiente horizontal e o 
gradiente vertical 
s�o calculados, e, a partir destes, s�o calculadas a magnitude e a orienta��o
de gradiente. Para cada \emph{pixel}, s�o armazenadas a magnitude e orienta��o
do canal de cor que tiver a maior magnitude naquele \emph{pixel}.

\subsection{C�lculo dos Histogramas}

A regi�o a ser descrita (ou, de maneira equivalente, as matrizes de magnitude e
                         orienta��o de gradiente) 
� particionada em uma malha de c�lulas de mesmo 
tamanho, e um histograma � calculado para cada c�lula, como demonstra a Figura
\ref{fig:partition_cell}. Em \cite{dalal2005histograms} s�o usadas $128$
c�lulas de $8 \times 8$ \emph{pixels}, em uma janela de $128 \times 64$
\emph{pixels}, por�m esse valor pode ser modificado de acordo com as
necessidades do problema a ser resolvido.

\begin{figure}[h]
  \centering
    \caption{Representa��o da parti��o em c�lulas.}
    \includegraphics[keepaspectratio=true,width=0.9\textwidth]
      {images/hist_cell.jpg}
  \fonte{Gil's \emph{CV blog},
      dispon�vel em:
        <https://gilscvblog.wordpress.com/2013/08/18/a-short-introduction-to-descriptors/>
          Acesso em jun. 2015.}
  \label{fig:partition_cell}
\end{figure}

As classes do histograma
representam faixas de �ngulos da orienta��o do gradiente e s�o ponderadas pela
magnitude dos mesmos. Para evitar artefatos de quantiza��o no histograma
obtido, cada gradiente � dividido entre as duas classes cujo limiar � o mais
pr�ximo do �ngulo do gradiente, ponderando pela diferen�a entre o �ngulo do 
gradiente e o limiar. A Figura \ref{fig:histseparable} representa tr�s exemplos
b�sicos da divis�o em classes, quando os gradientes est�o exatamente no centro
de uma classe ou exatamente na divis�o entre duas classes.

\begin{figure}[h]
  \centering
  \caption{(\ref{fig:gradwheel}) Tr�s gradientes e (\ref{fig:histcolor}) o 
    histograma respectivo obtido.}
  \begin{subfigure}[t]{0.35\textwidth}
    \includegraphics[width=\textwidth]{images/gradwheel.png}
    \caption{Gradientes}
    \label{fig:gradwheel}
  \end{subfigure}
  \begin{subfigure}[t]{0.55\textwidth}
    \includegraphics[width=\textwidth]{images/histcolor.png}
    \caption{Histograma}
    \label{fig:histcolor}
  \end{subfigure}
  \label{fig:histseparable}
  \fonte{Autoria Pr�pria}
\end{figure}

Na Figura \ref{fig:histsingle} est� representada a divis�o entre duas classes
quando a orienta��o do gradiente est� em uma posi��o qualquer.

\begin{figure}[h]
  \centering
  \caption{(\ref{fig:graddetail}) Um gradiente e (\ref{fig:gradsingle}) as duas 
    classes do histograma onde ele ser� adicionado.}
  \begin{subfigure}[b]{0.55\textwidth}
    \includegraphics[width=\textwidth]{images/gradsingle.png}
    \caption{Gradientes}
    \label{fig:gradsingle}
  \end{subfigure}
  \begin{subfigure}[b]{0.35\textwidth}
    \includegraphics[width=\textwidth]{images/graddetail.png}
    \caption{Histograma}
    \label{fig:graddetail}
  \end{subfigure}
  \label{fig:histsingle}
  \fonte{Autoria Pr�pria}
\end{figure}

As posi��es onde um gradiente ser� colocado podem ser obtidas em tempo
constante, associando a cada classe um identificador e
utilizando rela��es calculadas a partir do �ngulo do
gradiente e do n�mero de classes do histograma. A largura $|C|$ da faixa de 
valores de uma classe do histograma, sendo $n$ o
n�mero de classes do histograma, � dada pela equa��o \ref{eq:mod_c}.

\begin{equation} \label{eq:mod_c}
{|C|} = {{2 \pi} \over {n}}
\end{equation}

Associando a cada classe um identificador entre $0$ e $n-1$ ($a = 0$, $b = 1$,
                                                             $c = 3$,...),
           pode-se calcular em quais classes $id_{left}(G_{\theta})$ e
           $id_{right}(G_{\theta}$ atrav�s das equa��es \ref{eq:p_c1} e 
                       \ref{eq:p_c2}.

\begin{equation} \label{eq:p_c1}
id_{left}(G_{\theta}) = 
\left\lfloor {{G_{\theta}} \over {|C|}} - 0.5
\right\rfloor \bmod n
\end{equation}

\begin{equation} \label{eq:p_c2}
id_{right}(G_{\theta}) = \left( id (C_1) + 1 \right) \bmod n
\end{equation}

O valor de �ngulo limiar $\theta_{limiar}$ que divide as duas classes � obtido
pela equa��o \ref{eq:limiar}

\begin{equation} \label{eq:limiar}
  \theta_{limiar} = \left\lceil {{G_{\theta}} \over {|C|}} - 0.5 \right\rceil
  \cdot |C|
\end{equation}

A diferen�a entre $G_{\theta}$ e $\theta{limiar}$, representado por
$\Delta \theta$, � obtida pela equa��o \ref{eq:delta_theta}. Note que o valor
pode ser negativo.

\begin{equation} \label{eq:delta_theta}
  \Delta \theta = {G_{\theta} - \theta_{limiar}}
\end{equation}

Finalmente, os valores somados �s classes $id_{left}(G_{\theta})$ e
$id_{right}(G_{\theta})$, respectivamente denominados de $S_L$ e $S_R$, podem
ser calculados pelas equa��es \ref{eq:c1} e \ref{eq:c2}

\begin{equation} \label{eq:c1}
  S_L = \left( 0.5 - {{\Delta \theta} \over {|C|}}\right) \cdot G_{\rho}
\end{equation}

\begin{equation} \label{eq:c2}
  S_R = \left( 0.5 + {{\Delta \theta} \over {|C|}}\right) \cdot G_{\rho}
\end{equation}

Se o valor de \(I_{\phi}\) for menor do que \(\theta_{limiar}\), o valor de
\(\Delta \theta\) vai ser negativo e a maior parte do gradiente vai ser
colocada na classe $id_{left}(G_{\theta})$. Caso contr�rio, a maior parte ser�
colocada na classe $id_{right}(G_{\theta})$.
O valor de \(\Delta \theta\) sempre vai estar entre -0.5 e +0.5, sen�o o limiar 
mais pr�ximo estaria entre outras duas classes. 

\subsection{Gera��o de descritores HOG}

Os histogramas obtidos s�o agrupados em blocos de 
tamanho fixo e ent�o normalizados dentro de cada bloco.
O agrupamento em blocos funciona como uma janela deslizante sobre
as c�lulas, com cada histograma calculado possivelmente aparecendo
m�ltiplas vezes no descritor final. 

Em \cite{dalal2005histograms} s�o
apresentadas tr�s maneiras de normalizar os blocos, todas com desempenho
semelhante(para a tarefa de de localiza��o de pedestres). Dessas, foi escolhida
a \emph{L1-sqrt}, por ser computacionalmente mais leve.
Sendo $H$ um vetor contendo os valores de cada um dos
histogramas que comp�em o bloco, $H_i$ um elemento desse vetor e $\epsilon$ uma
constante pequena maior do que zero, a equa��o
\ref{eq:L1norm} demonstra o c�lculo da norma L1, denotada por $\|H\|$
e a equa��o \ref{eq:L1sqrt} a normaliza��o final resultante. 

\begin{equation} \label{eq:L1norm}
\|H\| = \sum_{i}H_i
\end{equation}

\begin{equation} \label{eq:L1sqrt}
H_i = \sqrt{{{H_i} \over {\|H\| + \epsilon}}}
\end{equation}

O descritor final � a concatena��o do
resultado de cada bloco, resultando em um vetor de \emph{n�mero de blocos
  \(\times\) 
n�mero de c�lulas por bloco \(\times\) n�mero de classes do histograma} 
dimens�es. Para a detec��o de pessoas, em \cite{dalal2005histograms}, foi
utilizada uma janela de $64 \times 128$ \emph{pixels}, com c�lulas de 
$8 \times 8$ \emph{pixels}, resultando numa malha de $8 \times 16$ c�lulas.
Para o agrupamento em blocos, foram utilizados blocos de tamanho $2 \times 2$,
     deslocando uma c�lula por vez, o que resulta numa malha de $7 \times 15 =
     105$
     blocos. O descritor final tem, ent�o, $105$ blocos, 
     com $4$ c�lulas por bloco, e um histograma de $9$ classes,
       totalizando em $105 \times 4 \times 9 = 3780$ dimens�es.



\chapter{CUDA: Arquitetura e Programa��o}\label{cap:recursos}

CUDA, \emph{Compute Unified Device Architecture}, � uma arquitetura de hardware 
e software, desenvolvida pela NVidia, que permite
programar uma GPU para executar c�digo de aplica��es gerais escrita em linguagens
de programa��o de alto n�vel.%\ref{http://www.nvidia.com.br/object/cuda_home_new_br.html}

CUDA nasceu como uma solu��o para os problemas do modelo GPGPU, onde a API gr�fica
era utilizada para computa��o de prop�sito geral. No modelo GPGPU, o programador 
deveria ter um grande conhecimento sobre o funcionamento da GPU e da API gr�fica.
Os problemas tinham que ser descritos como coordenadas de v�rtices, texturas e
\emph{shaders}, o que aumentava sua complexidade. Por �ltimo, GPUs n�o tinham
suporte a caracter�sticas b�sicas de programa��o, e a falta de suporte a vari�veis
de ponto flutuante de dupla precis�o restringia os modelos de programa��o que 
poderiam ser utilizados.
  

\section{GPU}

Uma Graphics Processor Unit (GPU) � uma arquitetura computacional especializada 
para tarefas com alta demanda computacional e altamente paraleliz�veis, devido ao
fato que GPUs possuem centenas de processadores, otimizados para executarem 
tarefas de forma paralela.

Na d�cada de 70, chips dedicados para acelerar o desenho de gr�ficos 
eram utilizados em placas-m�e de jogos de fliperama. Desde ent�o, esses chips
evolu�ram drasticamente suas fun��es e capacidade de processamento devido � demanda
por gr�ficos 3D em tempo real, eventualmente tornando-se as GPUs que conhe�emos atualmente.
Em 1986, a Texas Instruments lan�ou o primeiro microprocessador com uma GPU 
capaz de executar c�digo para aplica��es gerais. Por�m, apenas em 2001, se tornou mais pr�tico 
utilizar GPUs para aplica��es de uso geral.  

Atualmente, GPUs podem ser encontrada em computadores pessoais, laptops,
celulares e at� mesmo em placas embarcadas. Uma GPU discreta possui uma mem�ria 
RAM pr�pria e uma quantidade maior de n�cleos se comparada com GPUs integradas. 
No entanto, as GPUs discretas tamb�m s�o mais caras, sendo comumente utilizadas
para aplica��es gr�ficas e jogos eletr�nicos. As GPUs integradas s�o encontradas
junto a CPU na mesma pastilha e utilizam parte da mem�ria RAM do sistema
e n�o possuem tantos n�cleos quanto GPUs discretas, portanto possuem menor
capacidade computacional e s�o mais baratas. 

Uma GPU possui centenas de processadores especializados para opera��es paralelas e
com uma alta densidade de transistores. O n�cleo de uma GPU � dedicado � executar
um conjunto de instru��es �nico de maneira mais r�pida poss�vel, portanto sua 
arquitetura � muito mais simples se comparada � uma CPU que pode receber centenas
de conjuntos de instru��es.

%Figura CPU Style Cores x Slimming Down (GPU Cores)
%Fonte http://haifux.org/lectures/267/Introduction-to-GPUs.pdf

Os n�cleos partilham diferentes tipos de espa�os de mem�ria e saber qual mem�ria deve ser 
utilizada � pivotal para um bom desempenho. Registradores e mem�ria local s�o
apenas vis�veis para a linha de execu��o que escreve nessa mem�ria e os dados duram
enquanto essa linha de execu��o estiver ativa. A mem�ria local � um pouco mais
lenta do que os registradores.

A mem�ria compartilhada � vis�vel por todas as linhas de execu��o presentes em um
mesmo bloco, permitindo que essas linhas de execu��o comuniquem-se e troquem dados
entre si. A mem�ria global � vis�vel por todas as linhas de execu��o da aplica��o
e pelo host e os dados s�o mantidos pela dura��o da aloca��o de mem�ria feita
pelo host. 

Outros tipos de mem�ria s�o a mem�ria de texturas e a mem�ria de constantes. 
A mem�ria de constantes guarda todas as contantes utilizadas pela aplica��o,
efetivamente servindo apenas para leitura ap�s o inic�o da execu��o. A mem�ria
de textura guarda informa��es de texturas e tamb�m � utilizada como uma mem�ria
apenas para leitura. Ambas s�o beneficas para tipos especificos de aplica��es.

%Figura Tipos de mem�ria
%Fonte https://www.microway.com/hpc-tech-tips/gpu-memory-types-performance-comparison/


O funcionamento de uma GPU se resume a receber informa��es geom�tricas da CPU como
entrada e emitir uma imagem como sa�da. O processo se iniciar com a interface com a host,
onde a GPU ir� receber os comandos da CPU e carregar as informa��es de geometria
da mem�ria. O resultado desse primeiro passo ser� um conjunto de v�rtices com
informa��es associadas, como texturas, coordenadas, normais, entre outros.

O conjunto de v�rtices resultante da comunica��o com o host est� em fun��o do
espa�o do objeto, sendo necess�rio transform�-lo para ser exibido no espa�o da tela
do computador. Essa fase � chamada de processamento de v�rtices, onde todos os 
dados recebidos na etapa anterior s�o modificados para serem exibidos na tela. 
Nessa etapa n�o h� a cria��o ou o descarte de v�rtices, portanto todos os v�rtices 
recebidos da interface com o host s�o transformados.

Ap�s definir os locais dos v�rtices na tela, criam-se os tri�ngulos que ir�o 
preencher a tela. Antes de preencher a tela, tri�ngulos fora do campo de vis�o 
s�o rejeitados, sendo apenas utilizados os tri�ngulos que ir�o compor os pixels
da tela. Essa fase � chamada de configura��o de tri�ngulos. 

Ap�s inserir os tri�ngulos na tela, toda informa��o de texturas, normais, posi��o,
� utilizada para definir a cor final do pixel. Aqui s�o utilizadas t�cnicas como 
opera��es matem�ticas e mapeamento de texturas para colorir o pixel.Essa etapa � 
chamada de processamento de pixels.

Por �ltimo, a informa��o dos pixels � escrita para um framebuffer e a imagem
final � demonstrada na tela. A �ltima etapa � chamada interface de mem�ria. 

%Figura pipeline

\section{Arquitetura}

%ftp://download.nvidia.com/developer/cuda/seminar/TDCI_Arch.pdf
%http://docs.nvidia.com/cuda/cuda-c-programming-guide/#axzz3dtAz2c5O

A arquitetura CUDA foi desenvolvida pela NVidia com o objetivo de otimizar o uso de
GPUs para aplica��es de uso geral, utilizando c�digo escrito em linguagens de 
programa��o de alto n�vel. 

Uma GPU consiste de centenas de processadores em paralelo, interligados por interfaces 
de mem�ria muito mais r�pidas e complexas. 

Um n�cleo CUDA � um n�cleo de GPU criado para ser programado para aplica��es gerais. 
Esses n�celos seguem o paradigma 
SIMD (Single Instruction, Multiple Data), onde um grupo de processadores executam
a mesma instru��o de forma paralela, por�m cada processador � encarregado
de um pacote de dados diferente. 

GPUs utilizam a arquitetura SIMD (Single Instruction, Multiple Data), na qual 
a mesma instru��o � executada em diferentes dados. 


\section{Programa��o}

CUDA, uma \emph{framework} desenvolvida pela NVidia, permite que se utilize uma
GPU para processamento de prop�sito geral, tamb�m conhecido como GPGPU
(\emph{General-purpose computing on graphics processing units}). O
objetivo � que aplica��es possam utilizar a arquitetura paralela e velocidade
fornecidas por uma GPU para fins al�m de aplica��es gr�ficas.

A API permite organizar as threads, linhas de execu��o de programa que desenvolvem
atividades de forma concorrente, em blocos e criar conjuntos de
blocos chamados grids. � comum utilizarmos um n�mero muito maior de blocos nos grids
do que os fisicamente dispon�veis na placa, ent�o a API se encarrega de
realizar o escalonamento autom�tico das atividades entre os processadores e
blocos fisicamente existentes. 

Cada thread e bloco possui uma ID pr�pria, sendo poss�vel utilizar essa informa��o para que cada thread
trabalhe com um conjunto de dados �nicos. A API permite tamb�m organizar as threads em blocos e grids de maneira 2D ou 3D,
ficando o ID da Thread ou Bloco como uma tupla. Em processamento de imagens, as tarefas s�o usualmente realizadas em matrizes de duas 
dimens�es, portanto utilizar uma organiza��o em 2D facilita a escrita de c�digo, com cada Thread processando um pixel da imagem.

Sendo poss�vel programar e executar centenas de threads de forma �nica, tamb�m
� prov�vel que existam milhares de erros e bugs � serem consertados. Como o
programa principal e cada thread s�o executados de forma independente, n�o h�
uma falha ou aviso quando algo est� errado. � necess�rio esperar que as threads
terminem suas execu��es e sincroniz�-las com o programa para que possamos
procurar o erro. A API possui fun��es que permitem extrair qual o �ltimo erro
encontrado na execu��o e assim podemos eliminar bugs de nossas aplica��es.


\chapter{Implementa��o do HOG}

Foi decidido realizar a implementa��o do HOG como uma biblioteca, de maneira a 
facilitar a utiliza��o em projetos futuros. Foram inclusas na biblioteca uma
implementa��o em GPU e outra em CPU, com uma interface comum.

No restante deste cap�tulo primeiramente ser�o apresentadas as ferramentas de
\emph{software} e \emph{hardware} utilizadas, seguido da apresenta��o da
organiza��o da biblioteca. Ap�s isso, uma documenta��o breve de cada um
dos m�dulos e finalmente uma descri��o detalhada da implementa��o em GPU.

\section{Ferramentas utilizadas}

\subsection{Kit de desenvolvimento Jetson TK1}

A placa embarcada Jetson TK1, produzida pela NVidia, foi escolhida para o
projeto devido ao seu poder de processamento, baixo consumo e custo de \$192,
valor relativamente baixo se comparado com as alternativas dispon�veis no
mercado.

A placa utiliza o \sigla{SoC}{System on Chip}(\emph{System on Chip}) Tegra K1,
um processador de 4 cores ARM Cortex A15 e possui uma GPU integrada com a
arquitetura Kepler contendo 192 cores Cuda separados fisicamente em tr�s blocos.
Cada bloco tem sua pr�pria cache que � compartilhada entre seus 64 cores e pode
ser utilizada para sincronizar as threads em execu��o. A placa possui uma
mem�ria RAM de 2GB com largura de 64-bits, sendo ela compartilhada com entre a
CPU e a GPU, n�o sendo necess�rio copiar os dados entre as partes, diferente de
uma GPU discreta que possui sua pr�pria mem�ria dedicada separada da CPU. A
mem�ria � grande o suficiente para nos permitir trabalhar com imagem e v�deo.

Juntamente com a placa, foi utilizado o ambiente de desenvolvimento da NVidia,
maisespecificadamente, uma vers�o modificada do Eclipse que nos permite
codificar qualquer tipo de aplica��o para a placa e para GPUs.

\subsection{OpenCV}

OpenCV\footnote{http://opencv.org} � uma biblioteca de c�digo-aberto para aplica��es 
  de processamento de
imagem. A biblioteca � disponibilizada gratuitamente e recebe suporte por parte
de sua comunidade. Ela possui um m�dulo com algumas funcionalidades
implementadas para aplica��es em GPU e
a NVidia disponibiliza uma vers�o compilada com otimiza��es especif�cas para o 
processador Tegra K1, como parte do \emph{framework} CUDA.

Foi utilizada a vers�o 2.4.9 da biblioteca para desenvolvimento e a vers�o
otimizada da NVidia para execu��o na placa.

\subsection{TinyXML-2}

Para realizar a perman�ncia de configura��o do sistema, foi utilizada a
biblioteca TinyXML-2\footnote{http://grinninglizard.com/tinyxml2}, inclusa como
dois arquivos-fonte no projeto.

\subsection{Boost}

Alguns detalhes de implementa��o foram supridos pelo uso de algumas bibliotecas
do conjunto de bibliotecas Boost\footnote{http://boost.org}. Mais
especificamente, foram usadas as bibliotecas \emph{thread}, \emph{chrono},
  \emph{random} e \emph{filesystem} para, respectivamente, fazer chamadas
  ass�ncronas de fun��o, medir o tempo de execu��o de trechos de c�digo,
  gera��o eficiente de n�meros aleat�rios e automatiza��o de acesso aos
  arquivos do sistema para realiza��o dos testes.

\subsection{GitHub}

Foi utilizada a plataforma GitHub para realizar o versionamento de c�digo do
projeto. A biblioteca pode ser encontrada em: 
<https://github.com/EAPVA/GHoGLib>, o c�digo usado para testes em
<https://github.com/EAPVA/GHoGLib_Tests> e um c�digo de exemplo utilizando a
biblioteca em <https://github.com/EAPVA/GHoGLib_Example>.

\section{Organiza��o da biblioteca}

TODO

%A programa��o em uma GPU difere da programa��o em uma CPU em diversos aspectos
%como, por exemplo, a GPU executa m�ltiplos kernels em paralelo e tem um acesso
%especial � uma regi�o separada de mem�ria. Devido a sua arquitetura paralela,�
%poss�vel executar uma �nica tarefa com centenas de threads CUDA de maneira
%independente, utilizando dados de entrada diferentes para cada uma dessas
%threads. Portanto, caso seja possivel dividir uma tarefa grande em um
%subconjunto de tarefas menores e independentes, a programa�ao CUDA se
%demonstra eficiente.

\chapter{Avalia��o experimental}

Foram realizadas duas baterias de testes. A primeira comparando o desempenho de
tempo de execu��o do c�digo
escrito para GPU, o c�digo escrito para CPU, a implementa��o dispon�vel no
OpenCV para CPU e a implementa��o do OpenCV para GPU.

Para garantir a confiabilidade da solu��o utilizando GPU, tamb�m foi criada uma
bateria de testes comparando o resultado obtido na CPU e na GPU.

Todos os testes foram realizados utilizando otimiza��es, passando o par�metro 
$-O3$ para o compilador \emph{nvcc} e foram executados na placa Jetson TK1.

\section{Testes de desempenho de tempo de execu��o}

Cada teste de desempenho de tempo consiste em, para uma das implementa��es 
testadas, carregar uma imagem em alta resolu��o na mem�ria, escolher
aleatoriamente $1000$ janelas de tamanho fixo da imagem, medir e 
armazenar
o tempo gasto para executar a implementa��o em cada uma das janelas e
finalmente calcular a m�dia, desvio padr�o, m�ximo e m�nimo dos valores
medidos. Devido a limita��es de mem�ria e tempo de execu��o dos testes, o
n�mero de itera��es foi reduzido para $500$ nas janelas maiores do que $1280
\times 720$. Para a implementa��o deste trabalho � medido tamb�m o tempo
gasto para executar cada uma das tr�s fun��es da biblioteca necess�rias para
obter o descritor: normaliza��o de imagem, c�lculo dos descritores e c�lculo do
descritor.

Esse teste foi executado para cada uma das quatro implementa��es,
     com v�rios tamanhos de janela, de maneira a medir a
escalabilidade de cada uma. foram escolhidas $15$ janelas com tamanhos m�ltiplos
de $16 \times 9$ c�lulas (ou seja, tamanhos m�ltiplos de $128 \times 72$
                          \emph{pixels}).

\subsection{Resultados}

A Tabela \ref{tab:time1} cont�m os valores da m�dia de tempo gasto em uma
janela para cada uma das quatro implementa��es e a Figura \ref{fig:time1} um
gr�fico de barras correspondente.

\begin{table}[]
\centering
\caption{M�dia de tempo de execu��o para v�rios tamanhos de janela.}
\label{tab:time1}
\begin{tabular}{c|r|r|r|r}
 Janela & \multicolumn{1}{c|}{Implementa��o CPU} & \multicolumn{1}{c|}{Implementa��o
   GPU} & \multicolumn{1}{c|}{CPU OpenCV} & \multicolumn{1}{c}{GPU OpenCV} \\
     \hline
     128 x 72 & 2,97 & 3,82 & 1,05 & 12,28 \\ \hline
     256 x 144 & 11,71 & 5,15 & 4,31 & 20,80 \\ \hline
     384 x 216 & 26,52 & 7,81 & 10,25 & 25,36 \\ \hline
     512 x 288 & 47,56 & 11,60 & 19,82 & 22,90 \\ \hline
     640 x 360 & 74,32 & 16,47 & 31,20 & 29,40 \\ \hline
     768 x 432 & 107,85 & 20,02 & 46,88 & 25,59 \\ \hline
     896 x 504 & 147,37 & 24,01 & 64,45 & 31,71 \\ \hline
     1024 x 576 & 193,96 & 27,98 & 88,10 & 41,05 \\ \hline
     1152 x 648 & 244,01 & 33,42 & 105,22 & 50,80 \\ \hline
     1280 x  720 & 301,37 & 38,47 & 133,53 & 54,32 \\ \hline
     1408 x 792 & 364,07 & 41,79 & 159,66 & 63,15 \\ \hline
     1536 x 864 & 435,18 & 46,30 & 206,18 & 74,48 \\ \hline
     1664 x 936 & 510,34 & 47,83 & 225,29 & 82,69 \\ \hline
     1792 x 1008 & 590,94 & 46,81 & 270,39 & 96,65 \\ \hline
     1920 x 1080 & 678,01 & 47,41 & 301,78 & 122,82
  \end{tabular}
\end{table}

\begin{figure}[h!]
  \centering
  \caption{Compara��o de tempo de execu��o entre as quatro implementa��es}
  \includegraphics[keepaspectratio=true,width=0.9\textwidth]
    {images/Chart_AVG.png}
      \fonte{Autoria pr�pria.}
  \label{fig:time1}
\end{figure}

A Figura \ref{fig:time2} cont�m um gr�fico de barras obtido apenas com os
valores das implementa��es em GPU, de forma a facilitar a compara��o entre as
duas.

\begin{figure}[h!]
  \centering
  \caption{Compara��o entre as implementa��es do OpenCV e deste trabalho}
  \includegraphics[keepaspectratio=true,width=0.9\textwidth]
    {images/Chart_AVG2.png}
      \fonte{Autoria pr�pria.}
  \label{fig:time2}
\end{figure}

A Tabela \ref{tab:speedupcpu} mostra a acelera��o obtida (quantas vezes � poss�vel
executar um dos algoritmos na mesma quantidade de tempo que se executa o outro,
ou, em outras palavras, quantos quadros por segundo podem ser processados por
um enquanto o outro executa um quadro) pela implementa��o em GPU
em compara��o com a implementa��o em CPU deste trabalho.

\begin{table}[]
\centering
\caption{Acelera��o em compara��o com a implementa��o em CPU}
\label{tab:speedupcpu}
\begin{tabular}{c|r|r|r}
Tamanho & \multicolumn{1}{c|}{CPU} & \multicolumn{1}{c|}{GPU} &
\multicolumn{1}{c}{Acelera��o} \\ \hline
128 x 72 & 2,97 & 3,82 & 0,78 \\ \hline
256 x 144 & 11,71 & 5,15 & 2,27 \\ \hline
384 x 216 & 26,52 & 7,81 & 3,40 \\ \hline
512 x 288 & 47,56 & 11,60 & 4,10 \\ \hline
640 x 360 & 74,32 & 16,47 & 4,51 \\ \hline
768 x 432 & 107,85 & 20,02 & 5,39 \\ \hline
896 x 504 & 147,37 & 24,01 & 6,14 \\ \hline
1024 x 576 & 193,96 & 27,98 & 6,93 \\ \hline
1152 x 648 & 244,01 & 33,42 & 7,30 \\ \hline
1280 x  720 & 301,37 & 38,47 & 7,83 \\ \hline
1408 x 792 & 364,07 & 41,79 & 8,71 \\ \hline
1536 x 864 & 435,18 & 46,30 & 9,40 \\ \hline
1664 x 936 & 510,34 & 47,83 & 10,67 \\ \hline
1792 x 1008 & 590,94 & 46,81 & 12,62 \\ \hline
1920 x 1080 & 678,01 & 47,41 & 14,30
\end{tabular}
\end{table}

A Tabela \ref{tab:speedupocv} mostra a acelera��o obtida em compara��o com o
OpenCV.

\begin{table}[]
\centering
\caption{Acelera��o em compara��o com o OpenCV}
\label{tab:speedupocv}
\begin{tabular}{c|r|r|r}
Tamanho & \multicolumn{1}{c|}{OpenCV} & \multicolumn{1}{c|}{Implementa��o GPU} &
\multicolumn{1}{c}{Acelera��o} \\ \hline
128 x 72 & 12,28 & 3,82 & 3,21 \\ \hline
256 x 144 & 20,80 & 5,15 & 4,04 \\ \hline
384 x 216 & 25,36 & 7,81 & 3,25 \\ \hline
512 x 288 & 22,90 & 11,60 & 1,97 \\ \hline
640 x 360 & 29,40 & 16,47 & 1,78 \\ \hline
768 x 432 & 25,59 & 20,02 & 1,28 \\ \hline
896 x 504 & 31,71 & 24,01 & 1,32 \\ \hline
1024 x 576 & 41,05 & 27,98 & 1,47 \\ \hline
1152 x 648 & 50,80 & 33,42 & 1,52 \\ \hline
1280 x  720 & 54,32 & 38,47 & 1,41 \\ \hline
1408 x 792 & 63,15 & 41,79 & 1,51 \\ \hline
1536 x 864 & 74,48 & 46,30 & 1,61 \\ \hline
1664 x 936 & 82,69 & 47,83 & 1,73 \\ \hline
1792 x 1008 & 96,65 & 46,81 & 2,06 \\ \hline
1920 x 1080 & 122,82 & 47,41 & 2,59
\end{tabular}
\end{table}

A Tabela \ref{tab:timecpu} cont�m os valores da m�dia, desvio padr�o, m�ximo e
m�nimo para a implementa��o em CPU deste trabalho.

\begin{table}[]
\centering
\caption{Estat�sticas de tempo para a implementa��o em CPU deste trabalho}
\label{tab:timecpu}
\begin{tabular}{c|r|r|r|r}
Tamanho & \multicolumn{1}{c|}{M�dia} & \multicolumn{1}{c|}{Desvio Padr�o} &
\multicolumn{1}{c|}{M�nimo} & \multicolumn{1}{c}{M�ximo} \\ \hline
128 x 72 & 2,97 & 1,03 & 1,86 & 31,52 \\ \hline
256 x 144 & 11,71 & 1,68 & 9,49 & 45,09 \\ \hline
384 x 216 & 26,52 & 2,55 & 22,69 & 77,88 \\ \hline
512 x 288 & 47,56 & 3,56 & 41,32 & 104,21 \\ \hline
640 x 360 & 74,32 & 4,47 & 65,30 & 86,86 \\ \hline
768 x 432 & 107,85 & 5,78 & 97,51 & 124,53 \\ \hline
896 x 504 & 147,37 & 7,83 & 134,46 & 169,86 \\ \hline
1024 x 576 & 193,96 & 9,47 & 179,69 & 224,82 \\ \hline
1152 x 648 & 244,01 & 10,77 & 226,86 & 281,23 \\ \hline
1280 x  720 & 301,37 & 12,58 & 280,92 & 343,70 \\ \hline
1408 x 792 & 364,07 & 13,52 & 342,80 & 408,08 \\ \hline
1536 x 864 & 435,18 & 15,24 & 411,13 & 487,64 \\ \hline
1664 x 936 & 510,34 & 16,63 & 483,80 & 564,13 \\ \hline
1792 x 1008 & 590,94 & 18,80 & 562,46 & 564,13 \\ \hline
1920 x 1080 & 678,01 & 17,91 & 649,04 & 742,36
\end{tabular}
\end{table}

A Tabela \ref{tab:timegpu} cont�m os valores da m�dia, desvio padr�o, m�ximo e
m�nimo para a implementa��o em GPU deste trabalho.

\begin{table}[]
\centering
\caption{Estat�sticas de tempo para a implementa��o em GPU deste trabalho}
\label{tab:timegpu}
\begin{tabular}{c|r|r|r|r}
Tamanho & \multicolumn{1}{c|}{M�dia} & \multicolumn{1}{c|}{Desvio Padr�o} &
\multicolumn{1}{c|}{M�nimo} & \multicolumn{1}{c}{M�ximo} \\ \hline
128 x 72 & 3,82 & 0,16 & 3,64 & 5,96 \\ \hline
256 x 144 & 5,15 & 0,23 & 4,81 & 7,04 \\ \hline
384 x 216 & 7,81 & 0,59 & 2,59 & 22,43 \\ \hline
512 x 288 & 11,60 & 0,58 & 3,26 & 15,73 \\ \hline
640 x 360 & 16,47 & 0,39 & 3,26 & 20,26 \\ \hline
768 x 432 & 20,02 & 4,94 & 4,88 & 27,06 \\ \hline
896 x 504 & 24,01 & 7,22 & 6,36 & 42,82 \\ \hline
1024 x 576 & 27,98 & 9,95 & 10,58 & 46,06 \\ \hline
1152 x 648 & 33,42 & 12,59 & 12,79 & 57,82 \\ \hline
1280 x  720 & 38,47 & 14,80 & 14,52 & 73,69 \\ \hline
1408 x 792 & 41,79 & 15,57 & 17,17 & 72,18 \\ \hline
1536 x 864 & 46,30 & 16,73 & 20,38 & 79,30 \\ \hline
1664 x 936 & 47,83 & 17,28 & 22,98 & 118,59 \\ \hline
1792 x 1008 & 46,81 & 14,96 & 24,48 & 99,85 \\ \hline
1920 x 1080 & 47,41 & 13,41 & 25,62 & 111,37
\end{tabular}
\end{table}

A Tabela \ref{tab:timecpuocv} cont�m os valores da m�dia, desvio padr�o, m�ximo e
m�nimo para a implementa��o em CPU do OpenCV.

\begin{table}[]
\centering
\caption{Estat�sticas de tempo para a implementa��o em CPU do openCV}
\label{tab:timecpuocv}
\begin{tabular}{c|r|r|r|r}
Tamanho & \multicolumn{1}{c|}{M�dia} & \multicolumn{1}{c|}{Desvio Padr�o} &
\multicolumn{1}{c|}{M�nimo} & \multicolumn{1}{c}{M�ximo} \\ \hline
128 x 72 & 1,05 & 0,03 & 1,03 & 1,65 \\ \hline
256 x 144 & 4,31 & 0,85 & 4,15 & 24,62 \\ \hline
384 x 216 & 10,25 & 0,30 & 10,07 & 18,71 \\ \hline
512 x 288 & 19,82 & 0,10 & 19,46 & 20,55 \\ \hline
640 x 360 & 31,20 & 0,10 & 30,96 & 31,87 \\ \hline
768 x 432 & 46,88 & 0,39 & 46,60 & 58,27 \\ \hline
896 x 504 & 64,45 & 4,61 & 62,36 & 102,90 \\ \hline
1024 x 576 & 88,10 & 0,40 & 87,59 & 98,92 \\ \hline
1152 x 648 & 105,22 & 0,53 & 104,77 & 116,40 \\ \hline
1280 x  720 & 133,53 & 0,57 & 133,00 & 146,45 \\ \hline
1408 x 792 & 159,66 & 0,73 & 159,12 & 172,36 \\ \hline
1536 x 864 & 206,18 & 0,79 & 205,59 & 217,89 \\ \hline
1664 x 936 & 225,29 & 0,85 & 224,63 & 238,68 \\ \hline
1792 x 1008 & 270,39 & 0,94 & 269,62 & 284,55 \\ \hline
1920 x 1080 & 301,78 & 0,98 & 300,97 & 315,64
\end{tabular}
\end{table}

A Tabela \ref{tab:timegpuocv} cont�m os valores da m�dia, desvio padr�o, m�ximo e
m�nimo para a implementa��o em GPU do OpenCV.

\begin{table}[]
\centering
\caption{Estat�sticas de tempo para a implementa��o em GPU do openCV}
\label{tab:timegpuocv}
\begin{tabular}{c|r|r|r|r}
Tamanho & \multicolumn{1}{c|}{M�dia} & \multicolumn{1}{c|}{Desvio Padr�o} &
\multicolumn{1}{c|}{M�nimo} & \multicolumn{1}{c}{M�ximo} \\ \hline
128 x 72 & 12,28 & 1,00 & 7,12 & 20,17 \\ \hline
256 x 144 & 20,80 & 5,18 & 7,44 & 53,32 \\ \hline
384 x 216 & 25,36 & 5,51 & 9,75 & 49,26 \\ \hline
512 x 288 & 22,90 & 7,00 & 9,91 & 62,19 \\ \hline
640 x 360 & 29,40 & 10,17 & 13,50 & 90,31 \\ \hline
768 x 432 & 25,59 & 4,32 & 16,00 & 47,27 \\ \hline
896 x 504 & 31,71 & 6,72 & 22,01 & 79,91 \\ \hline
1024 x 576 & 41,05 & 6,41 & 28,90 & 59,90 \\ \hline
1152 x 648 & 50,80 & 9,98 & 36,43 & 97,16 \\ \hline
1280 x  720 & 54,32 & 5,80 & 45,48 & 74,55 \\ \hline
1408 x 792 & 63,15 & 7,72 & 54,73 & 87,95 \\ \hline
1536 x 864 & 74,48 & 1,98 & 64,16 & 94,58 \\ \hline
1664 x 936 & 82,69 & 3,95 & 74,59 & 99,38 \\ \hline
1792 x 1008 & 96,65 & 1,91 & 86,63 & 138,56 \\ \hline
1920 x 1080 & 122,82 & 15,10 & 99,05 & 158,41
\end{tabular}
\end{table}

A Tabela \ref{tab:timepartialcpu} cont�m as m�dias das medidas de tempo para cada uma das
etapas (normaliza��o, c�lculo dos gradientes, c�lculo do descritor) para a
implementa��o da CPU e a tabela \ref{tab:timepartialgpu} essas medidas para a
implementa��o da GPU. As tabelas \ref{tab:stddevcpu} e \ref{tab:stddevgpu}
cont�m, respectivamente, os valores de desvio padr�o para essas medidas.

% Please add the following required packages to your document preamble:
% \usepackage{multirow}
\begin{table}[]
\centering
\caption{Tempo gasto em cada uma das etapas na CPU.}
\label{tab:timepartialcpu}
\begin{tabular}{c|r|r|r|r}
Tamanho & \multicolumn{1}{c|}{Normaliza��o} & \multicolumn{1}{c|}{Gradientes} &
\multicolumn{1}{c|}{Descritor} & \multicolumn{1}{c}{Total} \\ \hline
\multirow{2}{*}{128 x 72} & 0,18 & 1,72 & 1,07 & 2,97 \\ \cline{2-5} 
& 5,99\% & 57,91\% & 36,10\% & 100,00\% \\ \hline
\multirow{2}{*}{256 x 144} & 0,72 & 6,63 & 4,37 & 11,71 \\ \cline{2-5} 
& 6,11\% & 56,59\% & 37,30\% & 100,00\% \\ \hline
\multirow{2}{*}{384 x 216} & 1,62 & 14,83 & 10,07 & 26,52 \\ \cline{2-5} 
& 6,10\% & 55,93\% & 37,97\% & 100,00\% \\ \hline
\multirow{2}{*}{512 x 288} & 2,88 & 26,18 & 18,50 & 47,56 \\ \cline{2-5} 
& 6,06\% & 55,05\% & 38,89\% & 100,00\% \\ \hline
\multirow{2}{*}{640 x 360} & 4,49 & 40,71 & 29,12 & 74,32 \\ \cline{2-5} 
& 6,05\% & 54,77\% & 39,18\% & 100,00\% \\ \hline
\multirow{2}{*}{768 x 432} & 6,46 & 58,66 & 42,72 & 107,85 \\ \cline{2-5} 
& 5,99\% & 54,39\% & 39,61\% & 100,00\% \\ \hline
\multirow{2}{*}{896 x 504} & 8,79 & 79,89 & 58,69 & 147,37 \\ \cline{2-5} 
& 5,96\% & 54,21\% & 39,83\% & 100,00\% \\ \hline
\multirow{2}{*}{1024 x 576} & 11,46 & 103,87 & 78,63 & 193,96 \\
\cline{2-5} 
& 5,91\% & 53,55\% & 40,54\% & 100,00\% \\ \hline
\multirow{2}{*}{1152 x 648} & 14,50 & 130,86 & 98,66 & 244,01 \\
\cline{2-5} 
& 5,94\% & 53,63\% & 40,43\% & 100,00\% \\ \hline
\multirow{2}{*}{1280 x  720} & 17,89 & 161,79 & 121,69 & 301,37 \\
\cline{2-5} 
& 5,94\% & 53,69\% & 40,38\% & 100,00\% \\ \hline
\multirow{2}{*}{1408 x 792} & 21,64 & 194,86 & 147,57 & 364,07
\\ \cline{2-5} 
& 5,94\% & 53,52\% & 40,53\% & 100,00\% \\ \hline
\multirow{2}{*}{1536 x 864} & 25,74 & 232,82 & 176,62 & 435,18
\\ \cline{2-5} 
& 5,92\% & 53,50\% & 40,59\% & 100,00\% \\ \hline
\multirow{2}{*}{1664 x 936} & 30,21 & 273,19 & 206,95 &
510,34 \\ \cline{2-5} 
& 5,92\% & 53,53\% & 40,55\% & 100,00\% \\ \hline
\multirow{2}{*}{1792 x 1008} & 35,09 & 316,22 & 239,63 &
590,94 \\ \cline{2-5} 
& 5,94\% & 53,51\% & 40,55\% & 100,00\% \\ \hline
\multirow{2}{*}{1920 x 1080} & 40,31 & 361,99 & 275,71 &
678,01 \\ \cline{2-5} 
& 5,95\% & 53,39\% & 40,66\% & 100,00\%
\end{tabular}
\end{table}

% Please add the following required packages to your
% document preamble:
% \usepackage{multirow}
\begin{table}[]
\centering
\caption{Tempo gasto em cada uma das etapas na GPU.}
\label{tab:timepartialgpu}
\begin{tabular}{c|r|r|r|r}
Tamanho & \multicolumn{1}{c|}{Normaliza��o} &
\multicolumn{1}{c|}{Gradientes} &
\multicolumn{1}{c|}{Descritor} &
\multicolumn{1}{c}{Total} \\ \hline
\multirow{2}{*}{128 x 72} & 0,87 & 0,88 & 2,07 & 3,82 \\
\cline{2-5} 
& 22,68\% & 23,12\% & 54,21\% & 100,00\% \\ \hline
\multirow{2}{*}{256 x 144} & 1,00 & 1,10 & 3,05 & 5,15
\\ \cline{2-5} 
& 19,36\% & 21,40\% & 59,24\% & 100,00\% \\ \hline
\multirow{2}{*}{384 x 216} & 1,33 & 1,53 & 4,95 & 7,81
\\ \cline{2-5} 
& 17,01\% & 19,60\% & 63,40\% & 100,00\% \\ \hline
\multirow{2}{*}{512 x 288} & 1,82 & 2,14 & 7,64 &
11,60 \\ \cline{2-5} 
& 15,68\% & 18,46\% & 65,85\% & 100,00\% \\ \hline
\multirow{2}{*}{640 x 360} & 2,43 & 2,92 & 11,12 &
16,47 \\ \cline{2-5} 
& 14,77\% & 17,75\% & 67,48\% & 100,00\% \\ \hline
\multirow{2}{*}{768 x 432} & 2,98 & 3,56 & 13,48 &
20,02 \\ \cline{2-5} 
& 14,88\% & 17,79\% & 67,33\% & 100,00\% \\ \hline
\multirow{2}{*}{896 x 504} & 3,61 & 4,29 & 16,10 &
24,01 \\ \cline{2-5} 
& 15,02\% & 17,89\% & 67,09\% & 100,00\% \\
\hline
\multirow{2}{*}{1024 x 576} & 4,16 & 4,96 &
18,86 & 27,98 \\ \cline{2-5} 
& 14,86\% & 17,73\% & 67,41\% & 100,00\% \\
\hline
\multirow{2}{*}{1152 x 648} & 4,91 & 5,99 &
22,51 & 33,42 \\ \cline{2-5} 
& 14,71\% & 17,94\% & 67,35\% & 100,00\% \\
\hline
\multirow{2}{*}{1280 x  720} & 5,37 &
6,85 & 26,25 & 38,47 \\ \cline{2-5} 
& 13,96\% & 17,80\% & 68,24\% & 100,00\%
\\ \hline
\multirow{2}{*}{1408 x 792} & 5,42 &
7,40 & 28,97 & 41,79 \\ \cline{2-5} 
& 12,98\% & 17,70\% & 69,32\% &
100,00\% \\ \hline
\multirow{2}{*}{1536 x 864} & 5,78 &
7,83 & 32,69 & 46,30 \\ \cline{2-5} 
& 12,48\% & 16,92\% & 70,61\% &
100,00\% \\ \hline
\multirow{2}{*}{1664 x 936} & 5,85 &
8,04 & 33,95 & 47,83 \\ \cline{2-5} 
& 12,22\% & 16,80\% & 70,98\% &
100,00\% \\ \hline
\multirow{2}{*}{1792 x 1008} & 5,94 &
7,91 & 32,95 & 46,81 \\ \cline{2-5} 
& 12,70\% & 16,90\% & 70,40\% &
100,00\% \\ \hline
\multirow{2}{*}{1920 x 1080} & 6,21
& 7,99 & 33,21 & 47,41 \\
\cline{2-5} 
& 13,11\% & 16,85\% & 70,04\% &
100,00\%
\end{tabular}
\end{table}

\begin{table}[]
\centering
\caption{Desvio padr�o para cada etapa da implementa��o em CPU}
\label{tab:stddevcpu}
\begin{tabular}{c|r|r|r}
Tamanho & \multicolumn{1}{c|}{Normaliza��o} & \multicolumn{1}{c|}{Gradientes} &
\multicolumn{1}{c}{Descritor} \\ \hline
128 x 72 & 0,01 & 0,38 & 0,93 \\ \hline
256 x 144 & 0,02 & 1,42 & 0,58 \\ \hline
384 x 216 & 0,04 & 2,46 & 0,26 \\ \hline
512 x 288 & 0,23 & 3,32 & 0,36 \\ \hline
640 x 360 & 0,07 & 4,25 & 0,37 \\ \hline
768 x 432 & 0,11 & 5,45 & 0,57 \\ \hline
896 x 504 & 0,12 & 7,26 & 0,86 \\ \hline
1024 x 576 & 0,12 & 8,57 & 1,34 \\ \hline
1152 x 648 & 0,14 & 9,79 & 1,44 \\ \hline
1280 x  720 & 0,15 & 11,39 & 1,64 \\ \hline
1408 x 792 & 0,34 & 12,14 & 1,86 \\ \hline
1536 x 864 & 0,19 & 13,55 & 2,13 \\ \hline
1664 x 936 & 0,20 & 14,65 & 2,43 \\ \hline
1792 x 1008 & 0,21 & 16,57 & 2,64 \\ \hline
1920 x 1080 & 0,22 & 15,54 & 2,82
\end{tabular}
\end{table}

\begin{table}[]
\centering
\caption{Desvio padr�o para cada etapa da implementa��o em GPU}
\label{tab:stddevgpu}
\begin{tabular}{c|r|r|r}
Tamanho & \multicolumn{1}{c|}{Normaliza��o} & \multicolumn{1}{c|}{Gradientes} &
\multicolumn{1}{c}{Descritor} \\ \hline
128 x 72 & 0,14 & 0,05 & 0,16 \\ \hline
256 x 144 & 0,11 & 0,06 & 0,14 \\ \hline
384 x 216 & 0,14 & 0,09 & 0,44 \\ \hline
512 x 288 & 0,22 & 0,13 & 0,37 \\ \hline
640 x 360 & 0,12 & 0,15 & 0,23 \\ \hline
768 x 432 & 0,65 & 0,93 & 3,50 \\ \hline
896 x 504 & 0,96 & 1,33 & 5,08 \\ \hline
1024 x 576 & 1,33 & 1,82 & 6,94 \\ \hline
1152 x 648 & 1,70 & 2,32 & 8,94 \\ \hline
1280 x  720 & 1,84 & 2,66 & 10,98 \\ \hline
1408 x 792 & 1,79 & 2,96 & 11,93 \\ \hline
1536 x 864 & 1,76 & 2,96 & 12,95 \\ \hline
1664 x 936 & 1,50 & 2,97 & 13,76 \\ \hline
1792 x 1008 & 1,30 & 2,70 & 11,83 \\ \hline
1920 x 1080 & 1,40 & 2,45 & 10,52
\end{tabular}
\end{table}


\vspace{7mm}

\subsection{Avalia��o dos resultados}

Em primeiro lugar pode-se notar que a acelera��o � maior para janelas de
tamanho maior, tanto para a implementa��o deste trabalho quanto para a
implementa��o em GPU do OpenCV. Para observar melhor esse resultado foi
calculado um valor de \emph{megapixels} processados por segundo, a partir dos
valores da tabela \ref{tab:time1}, de forma a ter um indicativo de velocidade
de execu��o para cada tamanho de janela. Esses valores est�o presentes na Tabela
\ref{tab:pixels} e um gr�fico correspondente na Figura \ref{fig:pixelspersec}. 
Nota-se
que a velocidade da implementa��o em CPU deste trabalho � aproximadamente
constante, a da implementa��o em CPU do OpenCV decresce com o
tamanho da entrada, provavelmente devido a \emph{overheads} no processamento de
muitos histogramas ou muitos blocos, a velocidade da implementa��o em GPU do
OpenCV cresce at� estabilizar em torno dos 19 \emph{megapixels} por segundo, e
a da implementa��o em GPU deste trabalho cresce continuamente at� o maior 
tamanho de janela usado.

\begin{table}[]
\centering
\caption{Velocidade de execu��o de cada implementa��o.}
\label{tab:pixels}
\begin{tabular}{c|r|r|r|r}
Tamanho & \multicolumn{1}{c|}{Implementa��o CPU} &
\multicolumn{1}{c|}{Implementa��o GPU} & \multicolumn{1}{c|}{CPU OpenCV} &
\multicolumn{1}{c}{GPU OpenCV} \\ \hline
128 x 72 & 3,10 & 2,41 & 8,75 & 0,75 \\ \hline
256 x 144 & 3,15 & 7,16 & 8,56 & 1,77 \\ \hline
384 x 216 & 3,13 & 10,62 & 8,09 & 3,27 \\ \hline
512 x 288 & 3,10 & 12,71 & 7,44 & 6,44 \\ \hline
640 x 360 & 3,10 & 13,99 & 7,39 & 7,84 \\ \hline
768 x 432 & 3,08 & 16,57 & 7,08 & 12,97 \\ \hline
896 x 504 & 3,06 & 18,81 & 7,01 & 14,24 \\ \hline
1024 x 576 & 3,04 & 21,08 & 6,70 & 14,37 \\ \hline
1152 x 648 & 3,06 & 22,34 & 7,09 & 14,69 \\ \hline
1280 x  720 & 3,06 & 23,95 & 6,90 & 16,97 \\ \hline
1408 x 792 & 3,06 & 26,68 & 6,98 & 17,66 \\ \hline
1536 x 864 & 3,05 & 28,66 & 6,44 & 17,82 \\ \hline
1664 x 936 & 3,05 & 32,56 & 6,91 & 18,83 \\ \hline
1792 x 1008 & 3,06 & 38,59 & 6,68 & 18,69 \\ \hline
1920 x 1080 & 3,06 & 43,74 & 6,87 & 16,88
\end{tabular}
\end{table}

\begin{figure}[h!]
  \centering
  \caption{Compara��o de velocidade entre as quatro implementa��es}
  \includegraphics[keepaspectratio=true,width=0.9\textwidth]
    {images/Chart_SPD.png}
      \fonte{Autoria pr�pria.}
  \label{fig:pixelspersec}
\end{figure}

Um resultado que pode ser obtido atrav�s dos dados das tabelas
\ref{tab:timecpu}, \ref{tab:timegpu}, \ref{tab:timecpuocv} e
  \ref{tab:timegpuocv} � que as implementa��es em GPU tendem a apresentar uma 
  varia��o maior de tempo de execu��o, especialmente considerando os valores 
  m�ximo e m�nimo.

Comparando as implementa��es em CPU e GPU deste trabalho, nota-se que a etapa
mais custosa na CPU � o c�lculo dos gradientes, que � a etapa que apresenta
maior acelera��o na implementa��o da GPU e que a etapa mais custosa
para a GPU � a terceira, c�lculo dos histogramas e normaliza��o em blocos.
Tam�m pode-se notar que a terceira etapa � a que apresenta maior varia��o de
tempo de execu��o na GPU. Isso se deve a problemas relacionados com acesso �
mem�ria compartilhada no c�lculo dos histogramas e com acesso � mem�ria 
global para agrupar os histogramas em blocos. 

Quando � realizada a soma parcial no \emph{kernel} do c�lculo dos histogramas o
acesso � mem�ria compartilhada � bem irregular, porque depende da
orienta��o dos gradientes sendo processados. Sendo assim, o acesso causa muitos
conflitos de banco imprevis�veis, serializando a execu��o do \emph{kernel} e
reduzindo a efici�ncia do \emph{kernel}. 

No agrupamento em blocos o acesso da mem�ria n�o � localizado, cada bloco
precisa de histogramas que est�o localizados em posi��es distantes um do outro.
Caso as transa��es fossem organizadas pela entrada, com cada \emph{thread}
sendo respons�vel por copiar uma classe do histograma para v�rias posi��es do
descritor final, ao inv�s de ser respons�vel por calcular um elemento do
descritor, a leitura poderia ser coalescida, mas a grava��o teria o problema de
acessar posi��es distantes de mem�ria.

\section{Testes de compara��o de resultado entre CPU e GPU}

Para comparar as duas implementa��es, foi realizado um procedimento semelhante
ao usado nos testes de desempenho de tempo. Uma imagem � carregada na mem�ria,
   s�o escolhidas $1000$ janelas de tamanho fixo ($1280 \times 640$
                                                  \emph{pixels}), e, para cada
   janela utilizada, s�o obtidos dois descritores, um utilizando a
   implementa��o em CPU e outro usando a implementa��o em GPU e � calculada a dist�ncia
   euclidiana entre os dois, normalizada pela soma das 
   magnitudes dos descritores, considerando cada descritor como um vetor
   \emph{n}-dimensional. Isso gera um valor de erro igual a $0$ caso os dois
   descritores sejam iguais e um valor igual a $1$ caso eles tenham uma
   orienta��o sim�trica, independente da magnitude de cada um.
   Finalmente, a m�dia, desvio padr�o, m�nimo e m�ximo s�o calculados para os
   $1000$ valores obtidos. O mesmo procedimento � repetido para a normaliza��o
   de imagem e para o c�lculo dos gradientes, considerando a concatena��o dos
   valores dos \emph{pixels} como um vetor, no caso da normaliza��o, e obtendo
   dois vetores, um com a concatena��o das magnitudes e outro com a
   concatena��o das orienta��es, para os
   gradientes. A Tabela \ref{tab:diff} cont�m os resultados desses
   testes.

\begin{table}[]
\centering
\caption{Distancia euclidiana normalizada entre os descritores da CPU e da GPU}
\label{tab:diff}
\begin{tabular}{c|r|r|r|r}
Tamanho & \multicolumn{1}{c|}{Normaliza��o} & \multicolumn{1}{c|}{Magnitude} &
\multicolumn{1}{c}{Orienta��o} & \multicolumn{1}{c|}{Descritor} \\ \hline
M�dia & 0,000 & 0,000 & 0,070 & 0,000 \\ \hline
Desvio Padr�o & 0,000 & 0,000 & 0,021 & 0,000 \\ \hline
M�nimo & 0,000 & 0,000 & 0,027 & 0,000 \\ \hline
M�ximo & 0,000 & 0,000 & 0,468 & 0,000
\end{tabular}
\end{table}

Com esse teste, se comprovou um erro de $0\%$ nos resultados, ou seja, os dois
descritores s�o equivalentes. As diferen�as obtidas para o c�lculo da
orienta��o do gradiente foram investigadas, e chegou-se a conclus�o de que a
fun��o usada para obter o arco tangente retorna, em alguns casos, $+\pi$ na CPU
e $-\pi$ na GPU e em alguns outros o contr�rio.
Esse resultado parcial � indiferente para o c�lculo dos histogramas.


%\chapter{Gest�o}

\section{Planilha de Horas}

\section{Planejamento de Riscos}

\input{plano/7_2_riscos.tex}

\section{Considera��es}


\chapter{Considera��es Finais}\label{cap:conclusao}



%---------- Referencias ----------

\bibliographystyle{abnt-num}
\bibliography{Referencias}




\end{document}
