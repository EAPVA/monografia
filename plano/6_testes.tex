\chapter{Avalia��o experimental}

Foram planejadas duas baterias de testes. A primeira comparando o desempenho de
tempo de execu��oo do c�digo
escrito para GPU, o c�digo escrito para CPU, a implementa��o dispon�vel no
OpenCV para CPU e a implementa��o do OpenCV para GPU.

Para garantir a confiabilidade da solu��o utilizando GPU, tamb�m foi criada uma
bateria de testes comparando o resultado obtido na CPU e na GPU.

Todos os testes foram realizados duas vezes, uma sem utilizar otimiza��es do
compilador, outra passando o par�metro $-O3$ para o compilador \emph{nvcc}.

\section{Testes de desempenho de tempo de execu��o}

Cada teste de desempenho de tempo consiste em, para uma das implementa��es 
testadas, carregar uma imagem em alta resolu��o na mem�ria, escolher
aleatoriamente $1000$ janelas quadradas de tamanho fixo da imagem, medir e 
armazenar
o tempo gasto para executar a implementa��o em cada uma das janelas e
finalmente calcular a m�dia, desvio padr�o, m�ximo e m�nimo dos valores
medidos.

Esse teste foi executado para cada uma das quatro implementa��es,
     com v�rios tamanhos de janela, de maneira a medir a
escalabilidade de cada uma.

\subsection{Resultados}

\section{Testes de compara��o de resultado entre CPU e GPU}

Para comparar as duas implementa��es, um banco de imagens com $1000$ arquivos 
foi montado, cada implementa��o foi executada em cada uma das imagens e foi
calculado, entre os dois vetores resultantes, a m�dia, desvio padr�o, m�ximo e
m�nimo de erro entre cada componente correspondente, al�m da dist�ncia 
euclidiana entre os dois vetores. Ap�s isso, a m�dia, desvio padr�o, m�ximo e
m�nimo dos resultados das $1000$ imagens foi calculado.

\subsection{Resultados}

