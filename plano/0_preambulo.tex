% ---------- Preambulo ----------
\instituicao{Universidade Tecnol\'ogica Federal do Paran\'a} % nome da instituicao
\programa{Departamento Acad\^emico de Inform\'atica} % nome do programa ou departamento
\area{Curso de Engenharia de Computa\c{c}\~ao} % �rea ou curso

\documento{Plano de projeto de Trabalho de Conclus\~ao de Curso} % [Trabalho de Conclus\~ao de Curso] ou [Disserta\c{c}\~ao] ou [Tese]
\nivel{Gradua\c{c}\~ao} % [Gradua\c{c}\~ao], [Especializa\c{c}\~ao], [Mestrado] ou [Doutorado]
\titulacao{Engenheiro} % [Engenheiro], [Tecn\'ologo], [Bacharel], [Mestre] ou [Doutor]

\titulo{\MakeUppercase{Extra��o Autom�tica de Placas de Ve�culos usando
  acelera��o por GPU}} % titulo do trabalho em portugues
\title{\MakeUppercase{Automatic License Plate Extraction using GPU acceleration}} % titulo do trabalho em ingles

\autor{Francisco Delmar Kurpiel} % autor do trabalho
\autordois{Luis Guilherme Camargo Machado}
\autortres{Marcelo Teider Lopes}
\cita{KURPIEL, Francisco D.; CAMARGO, Luis G. M.; LOPES, Marcelo T.} % sobrenome (maiusculas), nome do autor do trabalho

\palavraschave{ALPR, extra��o de placas veiculares, GPU, T-HOG, vis\~ao computacional} %substituir pelas palavras-chave relacionadas ao seu tema de pesquisa
\keywords{ALPR, license plate extraction, GPU, T-HOG, computer vision} % incluir palavras-chave em ingl�s

\comentario{\UTFPRdocumentodata\ de Engenharia da Computa\c{c}\~ao, apresentado ao \UTFPRprogramadata\ da \ABNTinstituicaodata\ como requisito parcial para obten\c{c}\~ao do t\'itulo de ``\UTFPRtitulacaodata\ em Computa\c{c}\~ao''.} %\\ \'Area de Concentra\c{c}\~ao: \UTFPRareadata.}

\orientador[Orientador:]{Prof. Bogdan Tomoyuki Nassu} % <- no caso de orientadora, usar esta sintaxe
%\coorientador[Co-orientadora:]{} % <- no caso de co-orientadora, usar esta sintaxe

\local{Curitiba} % cidade
\data{2015} % ano

\setcounter{tocdepth}{4} % para numera��o das se��es
\setcounter{secnumdepth}{4}

\numberwithin{equation}{chapter} %
%\numberwithin{table}{chapter} %
%\numberwithin{figure}{chapter} %

\makeatletter  %% this is crucial
 \renewcommand\subsubsection{\@startsection{subsubsection}{3}{\z@}%
                        {-2\p@ \@plus -0.5\p@ \@minus -0.5\p@}%
                        %{8\p@ \@plus 4\p@ \@minus 4\p@}%     <-- this is copied from the subsection command
                        {2\p@ \@plus 1\p@ \@minus 1\p@}%     <-- this is copied from the subsection command
                        {\normalfont\normalsize\bfseries\boldmath
                         \rightskip=\z@ \@plus 8em
 \pretolerance=10000 }}
\makeatother   %% this is crucial

%---------- Inicio do Documento ----------
\begin{document}

\capa % geracao automatica da capa

\folhaderosto % geracao automatica da folha de rosto
% dedicat�ria (opcional)
%\begin{dedicatoria}
% Texto da dedicat\'oria.
%\end{dedicatoria}

% agradecimentos (opcional)
%\begin{agradecimentos}
% Texto dos agradecimentos.
%\end{agradecimentos}

% epigrafe (opcional)
%\begin{epigrafe}
%\end{epigrafe}

%resumo
\begin{resumo}

Esse projeto tem o objetivo de extrair a localiza��o de placas de ve�culos de 
um \emph{stream} de v�deo, utilizando um algoritmo originalmente desenvolvido
para identificar se uma imagem � ou n�o texto.
 
Extra��o � uma das v�rias etapas necess�rias para o processo de 
reconhecimento autom\'atico de placas ve�culares. Em um \emph{stream} de v�deo, 
extra�mos, para cada \emph{frame}, a localiza��o de qualquer n\'umero de placas 
de ve\'iculos que possam ser encontradas. 

Como o algoritmo utilizado para a extra��o \'e altamente paraleliz\'avel, pois 
suas opera��es podem ser aplicadas de maneira independente em diversas 
regi\~oes da imagem, propomos sua implementa\c{c}\~ao em uma 
\sigla{GPU}{Graphical Processing Unit}.

A implementa��o ser\'a 
desenvolvida como uma biblioteca em \emph{C++} com o objetivo de ser utilizada 
em um sistema completo de reconhecimento de placas. 


\end{resumo}

%abstract
\begin{abstract}

This project aims to extract the location of license plates in a video stream,
by using an algorithm originally designed to identify if an image is or isn't 
text.

Extraction is one of the necessary steps for Automatic License Plate
Recognition. On a video stream we extract, for each frame, the location of any
number of license plates that can be found.

As the algorithm we're using for extraction is highly parallelizable, as it's
operations can be applied independently on several regions of the image, we
propose to implement it on a GPU.

The implementation will be developed as a C++ library, in
order for it to be used in a complete ALPR system.

\end{abstract}

% listas que recomenda-se a partir de 5 elementos
%\listadefiguras % geracao automatica da lista de figuras
%Elaborado de acordo com a ordem apresentada no texto, com cada item designado por seu nome espec�fico, acompanhado do respectivo n�mero da p�gina.

\listadetabelas % geracao automatica da lista de tabelas
%Elaborado de acordo com a ordem apresentada no texto, com cada item designado por seu nome espec�fico, acompanhado do respectivo n�mero da p�gina.

\listadesiglas % geracao automatica da lista de siglas
%Constitu�da de uma rela��o alfab�tica das abreviaturas e siglas utilizadas no texto, seguido das palavras ou express�es correspondentes grafadas por extenso. Utilizada apenas se houver siglas.

%\listadesimbolos % geracao automatica da lista de simbolos
%Elaborado de acordo com a ordem apresentada no texto, seguido do significado correspondente. Utilizada apenas se houver s�mbolos.

% sumario
\sumario % geracao automatica do sumario

%---------- Inicio do Texto ----------

\setcounter{page}{9} % *** Necess�rio arrumar manualmente antes de imprimir


