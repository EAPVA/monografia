\chapter{Cronograma}\label{cap:cronograma}

O tempo para cada atividade foi planejado de acordo com a experi�ncia dos
integrantes da equipe, tendo em conta o ciclo de desenvolvimento de uma semana
e deixando tempo de folga para a implementa��o para GPU e para a escrita da
monografia.

\begin{table}[h]

\caption{Cronograma de atividades}

\begin{center}

\begin{tabular}{|c|l|c|c|}
\hline
\textbf{ID} 
  & \textbf{Atividade}
  & \textbf{Depend�ncias} 
  & \textbf{Tempo previsto} \\ \hline

1 & Aplicativo de testes da biblioteca
  & -
  & 2 semanas               \\ \hline

2 & Implementa��o serial do \emph{T-HOG} 
  & 1 
  & 2 semanas               \\ \hline

3 & Estudo de CUDA e do \emph{kit} 
  & -
  & 4 semanas               \\ \hline

4 & Implementa��o do \emph{T-HOG} para GPU 
  & 1,2,3
  & 6 semanas               \\ \hline

5 & Valida��o da vers�o para GPU   
  & 4
  & 2 semanas               \\ \hline

6 & \emph{Tuning} e testes de desempenho 
  & 2
  & 4 semanas               \\ \hline

7 & Documenta��o e monografia                   
  & 5,6
  & 4 semanas               \\ \hline

\end{tabular}

\end{center}

\label{tab:cronograma}

\end{table}

