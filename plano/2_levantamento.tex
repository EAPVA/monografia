\chapter{Levantamento Bibiografico}\label{cap:levbibliog}

\section{Extra��o de caracter�sticas de imagens}

TODO

\section{Uso de GPU para Aplica��es Gerais}

TODO. Irei adaptar o texto do cap 4 para preencher aqui. 

\section{Uso de GPU para Extra��o de Caracter�sticas}

Em \cite{ready2007gpu} foram comparadas uma implementa��o baseada em CPU e
uma baseada em GPU(\emph{Graphical Processing Unit}) de um algoritmo
de rastreamento de caracter�sticas. Esse algoritmo recebe uma imagem da 
caracter�stica que deve ser encontrada e faz uma busca exaustiva na imagem,
comparando qual bloco da imagem possui uma semelhan�a maior com a caracter�stica
desejada. 
Os autores comparam a performance do c�digo em uma CPU 
Pentium IV, 3.2GHz com uma GPU Quadro NVS 285. Os resultados demonstram que a 
CPU, com uso de 90\% da sua capacidade, consegue 
detectar e acompanhar 40 caracter�sticas em um v�deo em tempo real enquanto que 
a GPU consegue seguir at� 500 caracter�sticas em tempo real. 

Em \cite{park2008low} � feita uma compara��o do desempenho de velocidade do algoritmo de 
detec��o de bordas de Canny aplicado em conjunto com o algoritmo \emph{Vector Coherence Mapping},
utilizado para detec�ao de movimentos em v�deos. Os algoritmos foram executados 
em uma CPU, 2.4 GHz Intel Core 2,  e as duas GPUs, 8800GTS-512 e uma placa de v�deo
\emph{onboard} 8600MGT do Apple MacBookPro. Tanto a implementa��o na CPU como 
na GPU, utilizaram os mesmos par�metros e estrutura para que as compara��es 
sejam as mais corretas poss�veis. Os resultados demonstram que a GPU \emph{onboard}
 obteve uma velocidade de execu��o 3.15 vezes melhor do que a CPU. A segunda GPU
obteve uma velocidade de execu��o 22.96 vezes melhor do que a CPU. Os autores demonstram
com os resultados que � poss�vel conseguir ganhos de performance significativos
ao se utilizar uma GPU para algoritmos pesados, contanto que o c�digo seja 
propriamente otimizado para tirar proveito da arquitetura de uma GPU.

