\chapter{Trabalhos relacionados}\label{cap:levbibliog}

\section{Extra��o de descritores de caracter�sticas de imagens}

TODO

\section{Uso de GPU para Extra��o de descritores de Caracter�sticas}

Em \cite{ready2007gpu} foram comparadas uma implementa��o baseada em CPU e
uma baseada em GPU(\emph{Graphical Processing Unit}) de um algor�tmo
de extra��o de caracter�sticas.
Os autores comparam a performance do c�digo em uma CPU 
Pentium IV, 3.2GHz com uma GPU Quadro NVS 285. Os resultados demonstram que a 
CPU, com uso de 90\% da sua capacidade, consegue 
seguir 40 caracter�sticas em tempo real enquanto que a GPU consegue seguir at� 
500 caracter�sticas em tempo real. 

Em \cite{park2008low} � feita uma compara��o da performance de um algoritmo de 
detec��o de bordas em tempo real sendo executado em uma CPU e em duas GPUs. A 
CPU utilizada foi uma 2.4 GHz Intel Core 2 e as GPUs utilizadas 
foram 8800GTS-512 e uma placa de v�deo \emph{onboard} 8600MGT do Apple 
MacBookPro. Tanto a implementa��o na CPU como na GPU, utilizaram os mesmos 
par�metros e estrutura para que as compara��es sejam as mais corretas 
poss�veis. Os resultados demonstram que a GPU \emph{onboard} obteve uma 
performance 3.15 vezes melhor do que a CPU. A segunda GPU
obteve uma performance 22.96 vezes melhor do que a CPU. Os autores demonstram
com os resultados que � poss�vel conseguir ganhos de performance significativos
ao se utilizar uma GPU para algoritmos pesados, contanto que o c�digo seja 
propriamente otimizado para tirar proveito da arquitetura de uma GPU.

