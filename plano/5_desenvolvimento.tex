\chapter{Desenvolvimento}

\section{Ferramentas utilizadas}

\subsection{Kit de desenvolvimento Jetson TK1}

A placa embarcada Jetson TK1, produzida pela NVidia, foi escolhida para o
projeto devido ao seu poder de processamento, baixo consumo e custo de \$192,
valor relativamente baixo se comparado com as alternativas dispon�veis no
mercado.

A placa utiliza o \sigla{SoC}{System on Chip}(\emph{System on Chip}) Tegra K1,
um processador de 4 cores ARM Cortex A15 e possui uma GPU integrada com a
arquitetura Kepler contendo 192 cores Cuda separados fisicamente em tr�s blocos.
Cada bloco tem sua pr�pria cache que � compartilhada entre seus 64 cores e pode
ser utilizada para sincronizar as threads em execu��o. A placa possui uma
mem�ria RAM de 2GB com largura de 64-bits, sendo ela compartilhada com entre a
CPU e a GPU, n�o sendo necess�rio copiar os dados entre as partes, diferente de
uma GPU discreta que possui sua pr�pria mem�ria dedicada separada da CPU. A
mem�ria � grande o suficiente para nos permitir trabalhar com imagem e v�deo.

Juntamente com a placa, utilizamos o ambiente de desenvolvimento da NVidia,
maisespecificadamente, uma vers�o modificada do Eclipse que nos permite
codificar qualquer tipo de aplica��o para a placa e para GPUs.

\section{OpenCV}

OpenCV � uma biblioteca de c�digo-aberto para aplica��es de processamento de
imagem. A biblioteca � disponibilizada gratuitamente e recebe suporte por parte
de sua comunidade. OpenCV possui uma vers�o dedicada para aplica��es em GPU e
tamb�m a NVidia disponibiliza otimiza��es especif�cas para o processador Tegra
K1.

%A programa��o em uma GPU difere da programa��o em uma CPU em diversos aspectos
%como, por exemplo, a GPU executa m�ltiplos kernels em paralelo e tem um acesso
%especial � uma regi�o separada de mem�ria. Devido a sua arquitetura paralela,�
%poss�vel executar uma �nica tarefa com centenas de threads CUDA de maneira
%independente, utilizando dados de entrada diferentes para cada uma dessas
%threads. Portanto, caso seja possivel dividir uma tarefa grande em um
%subconjunto de tarefas menores e independentes, a programa�ao CUDA se
%demonstra eficiente.
